\section{Diseño y configuración de la red de Comunicación}\label{ch:RedLan}
En este capítulo se detalla cómo es la estructura de comunicación del Robotario que permite dar instrucciones a los robots usando la localización y la navegación. La comunicación de los robots se lleva a cabo de manera inalámbrica. Para ello se ha elegido establecer la comunicación con los distintos robots mediante WiFi. Debido a que las placas de Arduino tienen una antena incorporada para la conexión a redes inalámbricas con la tecnología WiFI, es la opción más económica sin necesidad de añadir un módulo extra. Además, como el espacio donde se realizan los experimentos es limitado y de pequeña extensión está asegurada la cobertura.\\
 Para poner en situación el alcance de la comunicación, se hace un cálculo aproximado. Se tiene que la antena de la placa de Arduino tiene una PIRE de 18 dBm. La ganancia de la antena del router se desconoce, pero consultando el mercado se sabe que están en un valor desde 2 dBi a 9 dBi. Como se quiere probar el caso peor se toma 2 dBi, la sensibilidad del router es de 80 dBm y como frecuencia de referencia se toma 2.4GHz que es la correspondiente a la tecnología WiFi. Con estos datos y a partir de la ecuación de Friis\cite{cardama}
\begin{equation}
	Pr(dBm)=PIRE(dBm)+ Gr(dBi) + 20·log(\frac{\lambda}{4\pi d})
\end{equation} 
siendo Pr, la potencia recibida en bornes de la antena del receptor.
Despejando se obtiene $d$ que es la distancia máxima y da como resultado 994 metros. Hay que señalar que no se han considerado obstáculos, paredes, etc. Pero sirve para estimar aproximadamente el alcance. La localización actual del router permite una visión de los robots sin obstáculos por medio. Para el caso contrario, del router a la placa de Arduino, se tiene que la sensibilidad de Arduino es de 96 dBm, mucho mayor que la del router, por lo tanto, el caso anterior es el más restrictivo.\\
Las dimensiones del Robotario son adecuadas para esta comunicación, se recuerda que son 180cm de largo y 132 cm de ancho. El router se encuentra como máximo a tres metros de cualquier punto del Robotario.

\subsection{Red LAN}
\begin{figure}[h]
	\centering
	\includegraphics[width=0.7\textwidth]{CreacionInfraestructura/RedRobotario}
	\caption{Red LAN Robotario }
	\label{fig:ConfiguracionRed}
\end{figure}
Se tiene una red LAN con topología de red en estrella. En la figura~\ref{fig:ConfiguracionRed} se puede ver una representación gráfica de las conexiones, donde los robots se conectan de forma inalámbrica y el servidor está conectado mediante una conexión cableada, si bien puede ser también inalámbrica. En la situación actual del Robotario, se tienen 3 robots y el servidor\\
La transmisión de la información se realiza a través del protocolo UDP. Se usa este protocolo y no otro como TCP, para evitar retrasos en el envío de información. Un retraso puede provocar que, al dar una instrucción al robot esta no llegue a tiempo y se produzca un fallo en el algoritmo. Además, UDP es un protocolo más ligero al no mantener la conexión. Un factor a tener en cuenta es que los paquetes no van a llegar en orden o la información se  puede perder, pues no se garantiza la entrega de información con el protocolo UDP. Una solución ante este problema es enviar muchos paquetes en un intervalo pequeño de tiempo, teniendo cuidado con no congestionar la red. Esto puede ayudar en la comunicación servidor-robot. Como el servidor actualmente envía instrucciones al robot relacionadas con la posición, si el intervalo de envío es pequeño un paquete que llegue en desorden o se pierda, no afecta en gran medida, pues si el intervalo es más pequeño que la distancia avanzada el error es insignificante. En los datos que se requieran del robot, tales como velocidad, u otro sensor a bordo, basta con poner una marca de tiempo junto al envío de la información y ordenar los paquetes según van llegando al servidor.


La información que se envía mediante la red LAN tiene una estructura definida para que los robots o el servidor puedan identificar la información que les llega, de dónde les llega y cómo procesarla. Los paquetes tienen la siguiente cabecera:
\begin{itemize}
	\item \textbf{Id} identificador del origen
	\item \textbf{Op}. Código de operación, está en hexadecimal.
	\item \textbf{Len}. Longitud de los datos que se envían.
\end{itemize}
Cada robot dispone de un identificador que es un numero en decimal, el servidor también dispone de un identificador siendo este el 0. Las operaciones que se quieren ejecutar están almacenadas en una librería y codificadas en hexadecimal. Todos los robots y el servidor deben usar la misma librería de instrucciones y, por último, el apartado de la longitud sirve para detectar errores. Si llega un paquete con un  tamaño de información distinto al tamaño de la cabecera o a la longitud de la información que viene determinado por len y corresponde a datos enviados, el paquete recibido se desecha. Se muestra la estructura del mensaje enviado/recibido en la figura ~\ref{fig:estructuraMensaje} que comparten los robots y el servidor para poder comunicarse entre ellos. También se tienen definidas las instrucciones que se pueden ejecutar en el Robotario.\\
\begin{figure}[!h]
	 \begin{lstlisting}
		const int MAXDATASIZE =255; //numero de bytes que se pueden recibir
		const int HEADER_LEN = sizeof(unsigned short)*3;
		struct appdata{
			
			unsigned short id; //identificador
			unsigned short op; //codigo de operacion
			unsigned short len; /* longitud de datos */
			unsigned char data [MAXDATASIZE-HEADER_LEN];//datos		
		};
		//operacion error
		#define OP_ERROR            0xFFFF //Error de operacion
		#define OP_SALUDO           0x0001//Se verifica que el robot esta encendido y conectado a la red
		#define OP_MOVE_WHEEL       0x0002//da orden para mover ruedas wd,wi
		#define OP_STOP_WHEEL       0x0003//para las ruedas wd,wi
		#define OP_VEL_ROBOT        0X0004//devuelve la velocidad de las ruedas en rad/s wd,wi
		#define OP_IMU              0x0005//devuelve lectura de giroscopo y acelerometro
		#define OP_STOP_SERIAL      0X0006//para la comuniacion serie del robot
		#define OP_POSITION         0x0007//manda la posicion inicial de robot
		//broadcast
		#define OP_BROADCAST        0x9999//operacion de difusion

 \end{lstlisting}
 	\caption{Estructura de mensaje enviado/recibido y operaciones de instrucción }
 	\label{fig:estructuraMensaje}
 \end{figure}
La estructura appdata que aparece en la figura~\ref{fig:estructuraMensaje} corresponde a la de los paquetes que se acaba de comentar, donde se tiene un array para poder enviar datos, como velocidad  del robot, información del acelerómetro etc. Después de la estructura se pueden apreciar las instrucciones que se puede ejecutar. Las instrucciones se pueden ir ampliando y aumentar la complejidad del Robotario. El tamaño máximo de datos enviados consta de 255 bytes. Es un tamaño pequeño lo que dificulta la fragmentación del paquete debido a la unidad de transmisión máxima (MTU) de una red, que es el tamaño máximo permitido en bytes de un paquete que se puede transmitir en una red. Si se supera el tamaño máximo el paquete se fragmenta. La red actual tiene una MTU de 1500 bytes.
\newpage
\subsection{Calidad de servicio}
Se hace un análisis de los parámetros de la red para poder estimar sus límites y hacer una aproximación de la cantidad de dispositivos que se pueden conectar. 
\subsubsection{Ancho de banda}
 Se cuenta con un ancho de banda en la comunicación inalámbrica de 1750 Mbps, siendo esta la capacidad máxima de datos que se pueden transmitir a través del canal. Normalmente los valores reales son inferiores. Aun así, un valor algo inferior no es un problema para la comunicación del Robotario, pues se envían paquetes con una longitud máxima de 255 bytes más la cabecera UDP que son 8 Bytes y las demás cabeceras de encapsulación se tiene un tamaño máximo de 297 bytes. Comparando la longitud máxima de los paquetes con el ancho de banda, se observa que este es suficiente para transmitir la información con los 3 robots actuales, y se podria soportar más dispositivos en la red transmitiendo información.
\subsubsection{Latencia o retardo}
  La latencia se define como el tiempo que  tardan los flujos de datos en llegar a su destino. El retardo depende del tiempo de transmisión que se define como:
  \begin{equation}
 t_{t}=\frac{L}{C}
  \end{equation} 
  Donde L es la longitud máxima del paquete (297 bytes) y C se refiere al ancho de banda (1750Mbps), calculando se obtiene un tiempo de transmisión $t_{t}=1.36\mu s$. 
  El retardo debido a la congestión depende solo del router, y se produce por un tráfico alto de paquetes en el router o por como procesa éste la información.\\
\begin{figure}[!ht]
	\centering
	\includegraphics[width=0.5\linewidth]{CreacionInfraestructura/latencia}
	\caption{Captura de pantalla de una prueba de latencia de la red}
	\label{fig:latencia}
\end{figure}
En la figura~\ref{fig:latencia} se tiene la captura de una prueba de latencia de la red, esta prueba se ha realizado con 3 robots transmitiendo información al servidor y viceversa. Se ha realizado entre el servidor que está conectado por cable y una raspberryPi conectada a la red mediante WiFi. La prueba sirve para estimar la latencia. Se han hecho diversas pruebas, para encontrar el peor caso y como máximo se obtiene el resultado mostrado que es de 119 ms de retardo. Este tiempo máximo puede afectar de manera significativa en la transmisión de ordenes al robot, pero en general la latencia está por debajo de los 10ms que es un parámetro muy bueno y permite una transmisión de información de manera fluida.\\
\subsubsection{Jitter y perdida de paquetes}
\begin{figure}[!h]

	\centering
	\includegraphics[width=0.7\linewidth]{CreacionInfraestructura/jitterS}
	\caption{Analisis de la red con iperf3}
	\label{fig:jitters}

\end{figure}
Se realiza una prueba con iperf3 un programa que permite analizar las redes creando un servidor y cliente y enviando paquetes de longitud a elegir. Se puede elegir la velocidad de transmisión. En la prueba realizada que corresponde a la figura~\ref{fig:jitters} se ha elegido una velocidad de transmisión de 1 Mbps. La prueba se realiza con los robots y el servidor mandado paquetes a la red. En la figura~\ref{fig:jitters} se tiene un error en el parámetro de perdida de paquetes en el primer intervalo, esto se debe a que los paquetes han llegado desordenados y el programa ha contado mal el número de paquetes, es un problema reconocido de la aplicación. Después de este intervalo se corrige el error anterior y se tiene una pérdida de paquetes del 0\%. Este dato es muy bueno pues a pesar de estar enviando paquetes a la red procedentes de otros dispositivos no se ha perdido información esto significa que el router soporta el flujo de datos de manera adecuada.
El último parámetro a medir de la red es el jitter. Se trata de la variabilidad del retardo, se produce debido a la congestión de la red o a cambios en la ruta de los paquetes.
En la figura~\ref{fig:jitters} se puede apreciar que el máximo jitter ocurre en el primer intervalo de tiempo que es el intervalo erróneo. Después de este primer intervalo se puede ver como el jitter no pasa de los 2 ms, lo cual es un buen resultado.\\


Analizando  los parámetros de la red se tiene un buen resultado para los dispositivos conectados, de hecho, se podría conectar más sin problema. El factor que daría problemas dependiendo del número de dispositivos enviando información es la latencia, debido a como el router encamina y procesa la información y a que éste tiene una memoria limitada para almacenar los paquetes que le van llegando hasta que estos puedan ser enviados. Si se tienen muchos dispositivos enviando información en la red se pueden producir retardos debido a la memoria limitada del router.

