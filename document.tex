\documentclass[11pt, a4paper]{article} %tamaño mínimo de letra 11pto.
\usepackage{fancyhdr} % para personalizar la cabecera y el pie de pagina
\usepackage{lastpage} % para saber el numero de pagina del documento
\usepackage{graphicx} 
\usepackage{wrapfig}%figura en texto
\usepackage{caption}
\usepackage{subcaption}
\usepackage[breaklinks=true]{hyperref}
\usepackage[spanish]{babel} %Español 
\usepackage[utf8x]{inputenc} %Para poder poner tildes
\usepackage{vmargin} %Para modificar los márgenes
\usepackage{listings}
\usepackage{amsmath}%matrix
\usepackage{color}
\usepackage{listings}
\usepackage{import}
\lstset{ 
	language=C++,                % choose the language of the code
	basicstyle=\footnotesize,       % the size of the fonts that are used for the code
	numbers=left,                   % where to put the line-numbers
	numberstyle=\footnotesize,      % the size of the fonts that are used for the line-numbers
	stepnumber=1,                   % the step between two line-numbers. If it is 1 each line will be numbered
	numbersep=6pt,                  % how far the line-numbers are from the code
	backgroundcolor=\color{white},  % choose the background color. You must add \usepackage{color}
	showspaces=false,               % show spaces adding particular underscores
	showstringspaces=false,         % underline spaces within strings
	showtabs=false,                 % show tabs within strings adding particular underscores
	frame=single,           % adds a frame around the code
	tabsize=2,          % sets default tabsize to 2 spaces
	captionpos=b,           % sets the caption-position to bottom
	breaklines=true,        % sets automatic line breaking
	breakatwhitespace=false,    % sets if automatic breaks should only happen at whitespace
	escapeinside={\%*}{*)}          % if you want to add a comment within your code
}

\setmargins{2.5cm}{1.5cm}{16.5cm}{23.42cm}{10pt}{1cm}{0pt}{2cm}
%margen izquierdo, superior, anchura del texto, altura del texto, altura de los encabezados, espacio entre el texto y los encabezados, altura del pie de página, espacio entre el texto y el pie de página
% More defined colors
\usepackage[dvipsnames]{xcolor}

% Required package
\usepackage{tikz}
\usetikzlibrary{positioning}
\usepackage{circuitikz}

\begin{document}
	
%%%%%%Portada%%%%%%%
\begin{titlepage}
\centering
{ \bfseries \Large UNIVERSIDAD COMPLUTENSE DE MADRID}
\vspace{0.5cm}

{\bfseries  \Large FACULTAD DE CIENCIAS FÍSICAS} 
\vspace{1cm}

{\large DEPARTAMENTO DE XXXXX}
\vspace{0.8cm}

%%%%Logo Complutense%%%%%
{\includegraphics[width=0.35\textwidth]{logo_UCM}} %Para ajustar la portada a una sola página se puede reducir el tamaño del logo
\vspace{0.8cm}

{\bfseries \Large TRABAJO DE FIN DE GRADO}
\vspace{2cm}

{\Large Código de TFG:  [C\'odigo TFG] } \vspace{5mm}

{\Large [Título de TFG (exactamente el que aparece en la FICHA)]}\vspace{5mm}

{\Large [Título de TFG en ingl\'es (el que aparece en la FICHA)]}\vspace{5mm}

{\Large Supervisor/es: [Nombre del/os supervisores]}\vspace{20mm} 

{\bfseries \LARGE [Nombre del alumno]}\vspace{5mm} 

{\large Grado en Ingeniería Electrónica de Comunicaciones}\vspace{5mm} 

{\large Curso acad\'emico 20[XX-XX]}\vspace{5mm} 

{\large Convocatoria XXXX}\vspace{5mm} 

\pagestyle{myheadings} 
\end{titlepage}
\newpage

{\bfseries \large [Título extendido del TFG (si procede)] }\vspace{10mm} 

{\bfseries \large Resumen:} \vspace{5mm}

Esto es una prueba para probar el formato del Resumen. Esto es una prueba para probar el formato del ResumenEsto es una prueba para probar el formato del ResumenEsto es una prueba para probar el formato del ResumenEsto es una prueba para probar el formato del ResumenEsto es una prueba para probar el formato del ResumenEsto es una prueba para probar el formato del ResumenEsto es una prueba para probar el formato del ResumenEsto es una prueba para probar el formato del ResumenEsto es una prueba para probar el formato del ResumenEsto es una prueba para probar el formato del ResumenEsto es una prueba para probar el formato del ResumenEsto es una prueba para probar el formato del ResumenEsto es una prueba para probar el formato del ResumenEsto es una prueba para probar el formato del Resumen.
\vspace{1cm}

{\bfseries \large Abstract: }\vspace{5mm} 

This is a test to prove the abstract's layout.This is a test to prove the abstract's layout.This is a test to prove the abstract's layout.This is a test to prove the abstract's layout.This is a test to prove the abstract's layout.This is a test to prove the abstract's layout.This is a test to prove the abstract's layout.This is a test to prove the abstract's layout.This is a test to prove the abstract's layout.This is a test to prove the abstract's layout.This is a test to prove the abstract's layout.This is a test to prove the abstract's layout.This is a test to prove the abstract's layout.This is a test to prove the abstract's layout.This is a test to prove the abstract's layout.This is a test to prove the abstract's layout.This is a test to prove the abstract's layout.This is a test to prove the abstract's layout.This is a test to prove the abstract's layout.
\vspace{1cm}



{\Large\textbf{Extensión máxima 50 páginas sin contar portada ni resumen (sí se incluye índice, introducción, conclusiones y bibliografía}}
\newpage

%%Inicio: 
\tableofcontents
%SALTO DE PÁGINA
\newpage 
%SALTO DE PÁGINA 
%%%% Comenzar a contar inicio de páginas



\section{Introducci\'{o}n}


En la actualidad existen multitud de aplicaciones robóticas, en entornos industriales, en tareas que ayudan a los profesionales de la medicina, transporte de mercancias, incluso en la exploración espacial. Uno de los campos de interés de la robotica está relacionado con los sistemas multiagente, se trata de agentes individuales que colaboran entre ellos o interactúan entre ellos. Estos sistemas multiagente son utilizados para resolver tareas complejas, o tareas que requeriria mucho tiempo para un agente individual. En la robótica la aplicación de sistemas multiagente tiene actualmente uso en robots móviles que realizan tareas como exploración, transporte de mercancias, o usos para la guerra. Debido a que es un area de interés para la robotica, se han creado multitud de investigaciones que abordan el problema de la coordinación y colaboración entre los diferentes robots, estas investigaciones normalmente dan como resultado algoritmos que se aplican a los robots.\\
Los robots implicados en estas tareas suelen ser de un precio elevado dependiendo de las aplicaciones para las cuales se usen, por ello existe una necesidad poder comprobar los algoritmos mas alla de modelos computacionales y simulaciones en un entorno de prueba que no comprometa los prototipos relacioandos con las tareas .\\




\subsection{Estado del arte}
Respondiendo a la demanda de un entorno de pruebas , varios equipos de investigación han desarrollado entornos de prueba que permiten el análisis de algoritmos colaborativos o algoritmos que requieran el uso robots móviles, uno de los entornos de prueba que permite esta investigación y comprobación de los algoritmos es el Robotarium \cite{pickem2017robotarium}	\cite{8960572} en el cual se basa principalmente este proyecto. El Robotarium es un entorno de prueba desarrollado por un grupo de investigación del instituto  de Tecnología de Georgia en EEUU. El Robotarium fuel el primer entorno de pruebas que se puso a disposición del público, permitiendo que toda persona ya sea investigador, estudiante, trabajador o por simple afición, pudiera probar sus algoritmos en un entorno real de robots, se puede acceder a él en remoto, enviado los algoritmos los cuales son evaluados por la plataforma y te envían los resultados de los experimentos o  se puede ver en directo el experimento.\\

El Robotarium consta de más de un robot móvil, tienen un sistema de localización basado en cámaras cenitales actúan como un sistema de localización global y constan de un sistema de comunicación inalámbrica, además del entorno de pruebas, tienen desarrollado un simulador en matlab o python que permite comprobar que el algoritmo que se quiere implementar funciona en el Robotarium o se puede aplicar, además el coste de acceder es totalmente gratis lo cual lo hace muy atractivo. Actualmente se usa tanto para educación como para investigación.\\

Otro entorno de interés a mencionar es Duckietown \cite{7989179}, es un entorno desarrollado por el instituto Tecnológico de Massachusetts (MIT), y consta también de robots móviles pero a diferencia del Robotarium, los robots de Duckietown tienen cámaras incorporadas que permiten el reconocimiento del entorno. Duckietown esta enfocado en la educación y en la investigación. Tienen un curso enfocado principalmente a Duckietown para estudiantes de grado o postgrados, el cual permite al estudiante familiarizarse con los robots, el control de ellos y el reconocimiento del entorno, además debido a su gran documentación permite que cualquier institución que quiera tenerlo pueda adquirir su hardware y poder reproducir el entorno de Duckietown y realizar investigaciones sobre robots autónomos o inteligencia artificial.\\

Mi proyecto como se ha mencionado esta inspirado en el Robotarium, pero toma como referencia también Duckietown ya que los robots del robotario que es como he denominado mi entorno de pruebas de robots, tienen incorporadas cámaras a bordo que permiten el reconocimiento del entorno y permite desarrollar algoritmos que ademas de una localización global necesiten una localización local y reaccionar ante el entorno del robot. Los robots del robotario no son de un gran presupuesto y no se puede comparar a los entornos mencionados, pero hasta ahora se ha conseguido que permitan desarrollar algoritmos sencillos. Además el espacio del robotario, no es equiparable al Robotarium o Duckietown, pero permite tener al menos más de 2 robots, actualmente se tienen 3 montados con los cuales se pueden verificar diversos algoritmos relacionados con la robótica multi-agente.




\thispagestyle{empty}
\subsection{Motivacion}
Debido a la obligación de comprobar el correcto funcionamiento de los algoritmos de coordinacion, control o calobrativos desarrollados, más allá de las simulaciones, se necesita un entorno que permita observar el comportamiento de los agentes en un entorno lo mas realista posible y por lo tanto surge la necesidad de crear un banco de pruebas en el cual poder estudiar el comportamiento de los respectivos algoritmos.

\subsection{Objetivos}
El objetivo es crear un entorno de pruebas para poder probar algoritmos colaborativos y de coordinación u otros en los cuales sea necesario el uso de robots móviles. El entorno está basado en los trabajos anteriormente mencionados de Duckietown y Robotarium. Para poder llevarlo acabo se necesita cumplir los siguientes objetivos.
\subsubsection{{\large objetivos específicos }}
\begin{itemize}
	\item Construir y diseñar un robot capaz de realizar los algoritmos requeridos por el robotario.
	\item Diseñar un sistema de control que permita posicionar y mover el robot de acuerdo a las consignas recibidas desde los algoritmos de nivel superior.
	
	\item Configurar la comunicación de los robots, para poder comunicarse con otros robots y un servidor.
	
	\item Diseñar y configurar una red inalámbrica para la comunicación entre los robots y un servidor.
	
	\item Crear y configurar un servidor que estime la posición de los robots y se comunique con los diferentes robots del robotario.
	
	\item Construir y diseñar un sistema de localización, que identifique y localice a los robots involucrados en los experimentos.
	
	\item Llevar a cabo un experimento para demostrar el correcto funcionamiento del robotario.
\end{itemize}





%ya se ha hablado de robotario
\section{Hardware y software}
El robotario se ha podido desarrollar con los siguientes componentes.
\subsection{Robot móvil}

\begin{figure}[h]
	\centering
	\includegraphics[width=0.3\linewidth]{RobotAliexpres}
	\caption{Robot móvil}
	\label{fig:robotaliexpres}
\end{figure}

Se ha elegido un robot móvil de bajo presupuesto comprado en la pagina de Aliexpres\cite{robotAli} pero con piezas simples que se pueden reponer de manera sencilla si alguna se estropea, el chasis del robot incluye dos motores DC dos ruedas enganchadas a los motores, dos ruedas locas y consta de dos plantas.

\begin{figure}
	\centering
	\includegraphics[width=0.3\linewidth]{Arduinonano33}
	\caption{Placa de Arduino nano 33 IoT}
	\label{fig:arduinonano33}
\end{figure}
\subsection{Arduino nano 33 IoT}

Para tener un control sobre el robot, poder tener sensores en el robot y poder establecer la comunicación del robot con el servidor y los demas robots se usa una placa Arduino\cite{arduinoTienda}, Arduino es una compañia de software y hardware libre que desarrolla y diseña placas con microcontroladores, la placa Arduino adquirida cuenta con un microprocesador Arm® Cortex®-M0 32-bit SAMD2 además tiene incorporado una antena que permite la comunicación WiFi y Bluetooth, se ha adquirido en la tienda oficial de Arduino.
\subsection{L298n}
\begin{figure}[h]
	\centering
	\includegraphics[width=0.25\linewidth]{PuenteH}
	\caption{Puente H}
	\label{fig:puenteH}
\end{figure}
Para poder gobernar los motores correctamente se usa un puente H para cada motor. En este caso se usa un integrado el cual incorpora 2 puentes H y un regulador de tensión que proporciona 5 voltios a la salida y se aprovecha para alimentar la placa con microcontrolador. Se ha adquirido también en la pagina de Aliexpress \cite{puenteHAli}. Cuenta con  un rango de voltaje de entrada de 5V-35V. El puente H se alimenta con la tensión de entrada y se regula la alimetación que le llega al motor mediante una señal PWM\cite{PWM} procedente de la placa arduino. Esta señal PWM que proviene de arduino cuenta con 256 niveles, esto implica se cuantifica el voltaje de entrada en estos 256 voltajes, este numero límitado de niveles de voltaje provoca una limitación en el gobierno de los motores.
\subsection{Bateria}
\begin{figure}[h]
	\centering
	\includegraphics[width=0.3\linewidth]{ConstruccionRObot/bat}
	\caption{bateria}
	\label{fig:bateria}
\end{figure}
	Para alimentar el robot se usa una bateria LiFePo de 3 celdas de 3.3 V cada una en total se tiene 2200mAh. Se usan dos bateria en paralelo para poder tener una mayor intensidad y tener una duración mas larga de la beteria.

\subsection{Step-Down DC-DC}
	\begin{figure}[h]
		\centering
	\includegraphics[width=0.25\linewidth]{ConstruccionRObot/StepDown}
	\caption{Step-Down}
	\label{fig:StepDown}
	\end{figure}
Se tiene un conversor DC-DC para bajar el voltaje de salida de la bateria, en el capitulo ?? se comenta porque.También se ha adquirido en la tienda de Aliexpress[referencia]

\subsection{Encoder óptico}	%REVISAR LO DEL FILTRO%
	\begin{figure}[h]
	\centering
	\includegraphics[width=0.25\linewidth]{ConstruccionRObot/Encoder}
	\caption{Encoder óptico}
	\label{fig:encoder}
\end{figure} 
	Una de las partes fundamentales del robot es el encoder, sirve para poder medir la velocidad a la cual gira cada rueda y poder tambien estimar la distancia avanzada. El encoder óptico, es el modelo FC-03 , en la Figura ~\ref{fig:encoder} se pueder ver su forma. Del datasheet se sabe que tiene un LM393 el cual es un circuito integrado formado por dos comparadores, también lleva un optointerruptor,el cual consta de un diodo led y un transistor con la base expuesta, si este recibe luz, genera un pulso bajo de tensión y si no detecta luz genera un pulso alto de tensión, en conjunto con el comparador se obtendrán pulsos discretizados.\\

El encoder además tiene dos salidas, una analógica y otra digital, esta última es la que interesa debido a que es más útil para el microcontrolador leer entradas discretas. El voltaje de funcionamiento del sensor puede ser de 3.3V a 5V, como el microcontrolador solo soporta 3.3V será esta la alimentación que se proporcionará.\\ 
	Para contar pulsos de luz se usa una rueda ranurada que está enganchada al eje de la rueda,y  consta de 20 ranuras esto implica que se tiene una resolución de N=20 y se divide la rueda en $\Delta\theta=\frac{2\pi}{20} $ radianes.

\subsection{RaspberryPi y Picam}
	\begin{figure}[h]
	\centering
	\includegraphics[width=0.25\linewidth]{raspberry_cam}
	\caption{RaspberryPi conectada con la camara}
	\label{}
\end{figure}
	Se tiene una placa RaspberryPi4 montada en el robot para poder procesar imagenes adquiridas por una camara Picam que va conectada a la RaspberryPi.Se alimenta direactemente de la bateria usando un conversor DC-DC para adaptar el voltaje a 5 V y no tener que usar una segunda bateria para alimentar la RaspberryPi. La camara conectada a la raspberryPi permite obtener imagenes con un angular de 160º segun especificaciones del fabricante \cite{Picam}. Este modulo no es esencial para el robotario pero permite experimentar mas algoritmos debido a la localización con camara a bordo y vision artificial.
	
	\subsection{Camara logitech C920}
		\begin{figure}[h]
		\centering
		\includegraphics[width=0.3\linewidth]{Camaralogi}
		\caption{Cámara Logitech}
		\label{fig:camaralogi}
	\end{figure}

Se tienen 3 cámaras logitech que se usan para la localización de los robots. En el acutal montaje del robotario solo se usa una cámara. La conexión se realiza mediante USB al servidor.
\subsection{OpenCV}
	\begin{figure}[h]
	\centering
	\includegraphics[width=0.3\linewidth]{EjemploDeMarker}
	\caption{sacada de la documentación, ejemplo de un marker}
	\label{fig:ejemplodemarker}
\end{figure}

Para la visión por computador se usa una librería denominada OpenCV \cite{OpenCV} la cual está desarrollada en dos lenguajes de programación c++ y Python. Permite reconocer el entorno con una cámara y tratar las imagenes para desarrollar la aplicación que se desee. En particular yo uso la libreria ArUco\cite{ArUco} de OpenCV, que permite el reconocimiento de un marker de tamaño 4x4 formado por 16 bits.

	\subsection{Suelo de goma}	
Se usa un suelo de goma que permite que los robots no deslicen en el suelo, también sirve para deilimitar el área del robotario.


\clearpage\thispagestyle{empty}\cleardoublepage
\section{Metodología}\label{ch:metodologia}
Para poder llevar a cabo el robotario, se hace un estudio de los ejemplos ya existentes y se establece un método de trabajo.
\subsection{Selección de componentes}
Una vez que se ha recopilado información sobre lo necesario para elaborar el robotario se hace una lista de componentes a nivel de hardware y software.
	
\begin{center}
		\begin{enumerate}
		\item Robot móvil.
		\item Router.
		\item Ordenador que hace de servidor.
		\item Placa con microcontrolador(Arduino).
		\item Cámara webcam.
		\item Cámara a bordo de robot.
		\item Ordenador de placa simple(RaspberryPi).
		\item Selección de software para visión por computador.
		\item Selección de lenguaje de programación.
	\end{enumerate}
\end{center}
\subsection{Construcción de un primer robot}
Lo primero que se hace es evaluar el robot adquirido, como se comento antes, el robot cuenta con dos motores DC, se comprueba la zona de operación de los motores y se estima la velocidad a la cual pueden ir.
	\subsubsection{Prueba de encoders y estimación de la velocidad}
	El siguiente paso es verificar que los encoders ópticos adquiridos definen de manera adecuada los pulsos de salida y que la placa de Arduino cuenta de manera correcta los pulsos, para ello se configura la placa Arduino para que reaccione mediante interrupciones a los pulsos de voltaje proporcionados por el encoder óptico, es decir se provoca una interrupción cuando se tiene un nivel bajo de voltaje después de haber tenido uno alto y se crea una rutina sencilla que mide el tiempo entre interrupciones y cuenta las interrupciones que se producen, de esta manera se cuentan los pulsos.
	\begin{itemize} 
		{
		\item Se conecta la salida digital del encoder a la placa arduino y también a un osciloscopio de esta manera se puede comparar la lecturas que realiza arduino con las medidas reales de pulsos. Se obtiene que el microcontrolador de Arduino lee mas pulsos de los reales, esto es un problema pues no se puede estimar la posición ni la velocidad de manera correcta.\\
		\item Se diseña un filtro paso-bajo rc y se vuelve a comprobar la medición de los pulsos. Se encuentra un resultado favorable respecto al caso anterior, pero sigue habiendo errores en la medida que realiza el microcontrolador de Arduino.
		\item Además del filtro analógico, se diseña un filtro digital el cual consiste en un filtro paso bajo y un filtro de media móvil. Se obtiene un buen resultado, se tiene un error del 2\% en la cuenta de pulsos que realiza la placa de arduino.
		}
	\end{itemize}
	\subsubsection{Prueba de motor}
	
	 Se monta el robot y se ensambla el módulo L298N que incorpora dos puente H y un regulador de tensión. Se conecta la batería al módulo para alimentar los motores y la placa de Arduino mediante el regulador de tensión que proporciona 5 voltios a la salida, a su vez se conectan las salidas de la placa de arduino que proporcionan las señales PWM que se explican en el capitulo ~\ref{ch:ControlMotor}, para el puente H de cada motor y poder gobernar la velocidad a la cual se quiere que vaya ambos motores DC mediante la regulación del voltaje de entrada.\\
	Se encuentra con un problema y es que el puente H proporciona al motor un voltaje máximo de 9V procedente de la batería, haciendo que los motores tengan una velocidad máxima de 30 $rad/s$ y una velocidad mínima de 8.5 $rad/s$. Esto no interesa pues el espacio que se tiene es limitado y una velocidad alta del robot provoca que en un espacio reducido no se pueda apreciar los movimientos que realiza y no se pueda analizar los resultados.\\
	Se reduce el voltaje que alimenta el módulo L298n mediante un conversor DC-DC. Se regula para que proporcione un voltaje de alimentación de 6.5V. Al reducir el voltaje se consigue tener mas niveles de voltajes en rangos menores, debido a la reducción del voltaje se tiene una velocidad menor respecto al caso anterior, siendo la mínima velocidad 7.5$rad/s$ y la máxima 15 $rad/s$ con rozamiento. Se reduce la velocidad mínima porque se ha logrado una mayor resolución en rangos menores de voltaje.
	\subsubsection{programación de arduino y Comprobación del robot montado}
	
	 se crea el programa de control de velocidad y  de la comunicación entre dispositivos. La programación de la placa arduino se realiza con el lenguaje C++ y con las librerías que proporciona el entorno Arduino. El control se hace mediante un controlador PID para cada rueda el cual se detalla en el capitulo ~\ref{ch:ControlMotor}. Para verificar que está bien diseñado se realizan varias pruebas haciendo que ambos motores sigan la misma señal de referencia y al seguir la misma señal de referencia el robot debe ir en línea recta, si este se desvía sobrepasando un margen de error aceptable implica que se debe ajustar los parámetros del controlador.\\
	Se encuentran varios problemas.
	\begin{enumerate}
		\item Las ruedas del motor resbalan sobre el suelo donde se realizan las pruebas que son baldosas blancas estándar. Esto dificulta que el controlador actúe de manera correcta.
		\item Se encuentra un problema con la estima de la velocidad, la medida de velocidad oscila respecto al punto de referencia cuando las pruebas se realizan sobre el suelo, pero si las pruebas se realizan con las ruedas al aire se tienen lecturas que llegan a un estado estacionario.
		\item Las ruedas locas del robot, afectan a la orientación del robot debido al torque que se produce sobre ellas y a la fuerza de rozamiento. Esto es un problema que afecta al controlador de velocidad y a la dirección a la cual se quiere ir.
	\end{enumerate}



Para solucionar el primer problema, se elige un suelo de goma sobre el cual se van a realizar las pruebas, de esta manera las ruedas motrices no patinan y el movimiento se estabiliza.\\

Respecto al segundo problema se han localizado dos fuentes de origen, una es mecánica y otra es procedente de ruido eléctrico.\\
El problema mecánico se debe a que los ejes donde se enganchan las ruedas motrices tienen holgura y esto provoca una oscilación sobre el propio eje de la rueda motriz y otro problema es que el eje se dobla ligeramente cuando se posa sobre el suelo, esto último hace que haya mas superficie de contacto en un lado de la rueda que en otro, provocando perturbación en el movimiento. Estos problemas se deben a la calidad de los componentes y puesto que buscar y adquirir un componente adecuado o un mejor chasis del robot aumentaría el presupuesto se deja como esta y, se trata el problema mecánico como incertidumbre en la medida.\\
El ruido eléctrico se debe a que se alimenta el circuito de potencia y el de la placa de arduino con la misma batería, y lo único que separa un circuito del otro es el regulador de tensión del LN298n, que al ser de baja calidad, no consigue aislar un circuito del otro. La solución que se hace es colocar un condensador electrolítico, como desacoplo en la entrada de alimentación de la placa de arduino. A pesar de la solución se sigue obteniendo ruido debido a la diafonía que producen los pulsos lógicos de lectura de los encoders, ante esto no se puede hacer nada al menos en la placa de arduino de la que se dispone.\\
Las ruedas locas se cambian por otras ruedas, denominadas ruedas de bola que solucionan el problema.\\
Una vez que se tiene ajustado el robot, se construyen 2 robots mas.
\subsection{Configuración y desarrollo de red de comunicación}
El siguiente paso es configurar la red de comunicación, la cual como se ha comentado es necesariamente inalámbrica, por ello se selecciona un router que estaba disponible en el departamento, pero se encuentra un problema y es que este router es muy antiuguo y se debe buscar otro con unos parámetros mejores que el anterior, como el ancho de banda, potencia, procesamiento de la información, etc. En las placas de arduino se crea un programa que permite conectarse a la red LAN. A su vez se crea un pequeño servidor para realizar pruebas de comunicación.Se obtiene de las diversas pruebas realizadas, que se tiene una latencia baja para un dispositivo y el servidor, se realiza una prueba de envio de paquetes con mas arduinos involucrados y se obtiene una latencia mayor entorno a 100 ms, es un factor a tener en cuenta que puede afectar negativamente al robotario.Se elige el protocolo UDP para realizar el envío de información y evitar una latencia mayor y retrasos.\\
 Una vez que se tiene todo funcional se desarrollan funciones, protocolos e instrucciones de comunicación entre las placas de arduino y el servidor de manera que se pueda producir una correcta comunicación. Toda la programación se realiza en C++ y C.\\
 Además de la comunicación inalámbrica se crea una comunicación serie a traves del puerto USB entre arduino y una RaspberryPi que va implementada en el robot. Se usa la misma estructura utilizada para la comunicación WiFi, pero se envia bit a bit.

\subsection{Programación para visión por computador}
Una vez que se tiene la comunicación y el robot preparado se realiza un estudio de la librería de OpenCV en especial de la librería ArUco perteneciente a OpenCV.\\
Estudiada la librería de OpenCV se crea un programa que permite localizar los Markers en el mundo visto por la cámara que en este caso es la cámara logitech, transformando los pixeles de la imagen en coordenadas X,Y,Z con distancia medida en metros, esto realiza con una calibración de la cámara y obteniendo dos matrices, una con los parámetros de distorsión de la cámara y  otra con la relación pixeles metros.\\
 Una vez que se tiene configurado el reconocimiento de markers y su localización en el mundo de la cámara, se procede a hacer lo mismo con la cámara de la raspberryPi de esta manera se tienen calibradas los dos tipos de cámaras.\\
 Finalizada la calibración de las cámaras se colocan en su posición final, la de la raspberryPi se coloca en el robot y la cámara Logitech se coloca encima del lugar que se tiene reservado para el robotario, haciendo de cámara cenital.\\ 
 Por simplicidad solo se usa una cámara cenital.
\subsection{Configuración y programación de RaspberryPI}
Con la cámara de la raspberryPi configurada, se desarrolla un programa que reconoce el entorno y dependiendo del algoritmo a desarrollar, manda instrucciones a la placa de arduino mediante el puerto serie. A su vez se configura la raspberryPi para poder comunicarse con el servidor y poder enviar datos en tiempo real de posición y velocidad, además estos datos se guardan en un archivo en la raspberryPi para poder procesar los datos de los experimentos.

\subsection{Montaje final de Robotario}
 Se coloca una cámara cenital que sirve como sistema de localización global, la cámara se coloca encima del los límites del robotario y lo mas centrada posible.
Por último se instala un suelo de goma en la zona del robotario que maraca los límites. Se hacen pruebas de movimientos con los robots y se detecta que a pesar de ser el suelo de goma, al ser de baja calidad los robots se quedan atrapados en unas zonas, reduciendo su velocidad o en otras zonas sus ruedas motrices resbalan. La solución a futuro es adquirir un suelo de goma duro y liso, similar al que te puedes encontrar en un gimnasio.

En la figura ~\ref{fig:montajefinal} se puede ver como queda el robotario montado, con 3 robots. arriba se puede ver la cámara colocada. En la imagen se aprecia mejor como el espacio es pequeño, por ello se tiene la necesidad de que los robots vayan mas despacio.
\begin{figure}
	\centering
	\includegraphics[width=0.6\linewidth]{MontajeFinal}
	\caption{Montaje final del robotario}
	\label{fig:montajefinal}
\end{figure}
\clearpage\thispagestyle{empty}\cleardoublepage
\section{Estimación de la velocidad}
La estimación de la velocidad se realiza con encoders ópticos, estos detectan con ayuda de una rueda ranurada el paso de luz o no mediante un fotodiodo que incorpora el encoder y un transistor con la base expuesta que reacciona a la luz, a la salida devuelven un valor alto de voltaje si no se ha detectado luz o un valor bajo de voltaje si se ha detectado luz, en la Figura ~\ref{fig:encoderrobot} se puede apreciar la rueda ranurada y como está colocado el encoder en el robot. Los encoders como tal no pueden contar pulsos o medir la velocidad, esto se hace con ayuda de la placa de Arduino, la cual se ha configurado para que actúe mediante interrupciones a las variaciones de voltaje que proporciona la salida digital del encoder, el porque se hace con interrupciones es para no mantener el procesador de la placa Arduino ocioso estimando la velocidad. 
		\begin{figure}[h]
	\centering
	\includegraphics[width=0.45\linewidth]{ConstruccionRObot/EncoderRobot}
	\caption{Colocación del encoder}
	\label{fig:encoderrobot}
\end{figure}


La manera en que se estima la velocidad es midiendo el intervalo de tiempo entre interrupciones y en la misma rutina de interrupción se cuentan los pulsos, de esta manera se puede medir la distancia recorrida por cada rueda y estimar la velocidad de cada rueda. La rueda ranurada como se ha comentado antes, consta de 20 ranuras,  esto implica que se tiene una resolución de N=20 y se divide la rueda en $\Delta\theta=\frac{2\pi}{20} $ radianes, conociendo las dimensiones de la rueda, la cual tiene un radio de $3.35 cm$ se puede conocer la distancia que avanza cada rueda y por lo tanto la distancia recorrida por el robot, la mínima distancia que se puede medir es $3.35 ·\frac{2\pi}{20}=1.05$ cm es una resolución suficiente para los motores que se tienen ya que por más precisión que se tenga en la medida como se ha comentado antes la mínima velocidad alcanzable es de $7.5 rad/s$ que conociendo las dimensiones del rueda equivale a $25.13 cm/s$, esta es la mínima velocidad de giro de la rueda y como se comento antes para un voltaje de 6.5 voltios del motor que es el voltaje maximo impuesto por mi, se obtiene una velocidad máxima de $15rad/s$  esta velocidad es la que va a condicionar mi periodo de muestreo pues como un $rad/s= \frac{2·\pi}{20·Ts}$ donde $Ts$ es el tiempo entre pulsos, despejando se obtiene que $Ts=20.96 ms$ es decir que mi tiempo de muestreo a partir del teorema de Nyquist debe de ser al menos menor que la mitad de este tiempo. Yo elijo un tiempo 3 veces menor es decir de $5 ms$ siendo este tiempo mínimo que he estimado que me permite Arduino tener corriendo su programa y realizando los cálculos correspondientes.








\clearpage\thispagestyle{empty}\cleardoublepage
\subsection{Tratamiento del ruido}

Las lecturas que realiza la placa de Arduino de los pulsos de los encoders ópticos no son buenas, la placa de Arduino lee más pulsos de los reales, esto provoca que la estimación de la velocidad no se haga de manera correcta.
Se tienen dos problemas identificados, uno es mecánico y se debe a que los ejes donde se enganchan las ruedas motrices tienen holgura y esto provoca  una oscilación sobre el propio eje de la rueda motriz y otro problema es que el eje se dobla ligeramente cuando se posa sobre el suelo. Esto último hace que haya mas superficie de contacto en un lado de la rueda que en el otro. Y Estos problemas provocan oscilaciones en el movimiento y por lo tanto en las lecturas. Estos problemas se deben a la calidad de los componentes puesto que es un robot de bajo presupuesto no se tiene solución a no ser que se cambie de hardware. Se trata este ruido como incertidumbre en la medida.\\

El otro problema identificado es ruido eléctrico. Se debe a que se alimenta el circuito de potencia y el de la placa de arduino con la misma batería, y lo único que separa un circuito del otro es el regulador de tensión que incorpora el integrado LN298n. Al ser de baja calidad no consigue aislar un circuito del otro. Se coloca un condensador de desacoplo en la alimentación de la placa de arduino para minimizar el ruido de alta frecuencia. A pesar de está solución debido a la comunicación de la placa de Arduino con los motores y con los encoders, que se hace mediante pulsos. Se tiene ruido debido a la diafonía que producen estas señales, se ha trenzado los cables como medida de protección pero, en el propio integrado de la placa de Arduino no se puede hacer nada para solucionarlo.\\

\begin{figure}[h]
	\centering
	\includegraphics[width=0.7\linewidth]{estimaVelocidad/TratamientoRuido/PulsosEncoder_sinFiltro}
	\caption{Salida de encoder sin filtro}
	\label{fig:pulsosencodersinfiltro}
\end{figure}
En la figura ~\ref{fig:pulsosencodersinfiltro} se tiene una lectura del encoder que se hace con un osciloscopio. Se puede apreciar que existe un pulso que no debería estar entre el segundo -0.35 y el segundo -0.3, además de los diferentes niveles de voltaje intermedios que se aprecia en la figura, esto provoca errores en la lectura del encoder.\\

	

 Una solución que se aplica es diseñar un filtro paso-bajo, la manera mas sencilla de hacerlo es mediante un condensador, este condensador se coloca entre la salida analógica y tierra. Se usa condensador  de 47 $n_{f}$ este filtrará componentes de alta frecuencia y no deformará el pulso de manera significativa. Hay que modificar las interrupciones para que se hagan cuando la placa de Arduino detecte un flanco de subida. En la figura ~\ref{fig:pulsosencoderconfiltro} se puede observar el resultado final con el filtro paso-bajo, se han eliminado pulsos intermedios y se tiene definido un flanco de subida que se utiliza para las interrupciones.\\
 		\begin{figure}[h]
 	\centering
 	\includegraphics[width=0.6\linewidth]{estimaVelocidad/TratamientoRuido/PulsosEncoder_ConFiltro}
 	\caption{Lectura de pulsos con filtro paso-bajo}
 	\label{fig:pulsosencoderconfiltro}
 \end{figure}
 
A pesar de que este método corrige de buena manera el problema con el ruido y la estima de la velocidad, la placa de Arduino sigue contando más pulsos de los reales, esto se puede deber al problema de diafonía que se comentó o a la interferencia causada por el campo magnético que generan los motores DC, que están muy cerca de los encoders y de la placa de Arduino. Se ha alejado lo máximo posible la placa Arduino de los motores DC, aun así se sigue teniendo error en la cuenta de pulsos que realiza la placa de Arduino, por lo que se recurre a filtros digitales que irán programados en la rutina de la placa de Arduino que se usa para estimar la velocidad. 

Uno de los métodos aplicados, es poner un límite de tiempo entre detecciones de pulsos, se analiza con un osciloscopio el periodo de la señal a una velocidad máxima correspondiente al voltaje máximo impuesto, las medidas se realizan con las ruedas al aire sin el rozamiento del suelo. De este modo los valores obtenidos son mayores que los descritos en la sección~\ref{ch:controlmotorLimitacion}. También se hace la condición menos restrictiva  al no tener en cuenta el rozamiento del suelo, e impide que si por alguna razón la rueda gira más rápido de $15 rad/s$ no se obtengan errores en la medida, y se tenga una lectura correcta y el controlador actúe correctamente. Del osciloscopio se obtiene que este periodo es de 13.54ms en la Figura ~\ref{fig:maxvelencoder} se puede ver los pulsos de los encoders para una velocidad máxima que corresponde con un voltaje de 6.5V. Con esta configuración se condiciona la detección por interrupciones de arduino, despreciando los pulsos que ocurren en un intervalo inferior de tiempo.\\

	\begin{figure}[h]
	\centering
	\includegraphics[width=0.7\linewidth]{estimaVelocidad/TratamientoRuido/maxVelEncoder}
	\caption{Pulsos de encoder a velocidad máxima de la rueda. \\ Voltaje de 6.5V en el motor}
	\label{fig:maxvelencoder}
\end{figure}

Esta condición de tiempo para contar pulsos ha mejorado notablemente respecto a los casos anteriores, comparando los pulsos contados con la placa de Arduino y los que se obtienen con el osciloscopio, de 500 pulsos se obtiene un error del 5\% aproximadamente, pero debido a los problemas mecánicos mencionados, se sigue teniendo ruido en la lectura de pulsos, por ello como último recurso se aplica un filtro de media móvil que hace un promedio de los intervalos de tiempo entre dos pulsos consecutivos que se van registrando. Así cuando la rutina de Arduino tome un valor para estimar la velocidad, tomara un promedio de dichos valores.
\[ 
y(i)=\frac{1}{M} \sum_{j=0}^{M-1}x(i+j)
\]
El filtro  hace una media de los últimos M valores de tiempo que se van registrando, donde M se denomina la ventana del filtro, de esta manera se consigue una lectura más suave de la velocidad. Se ha probado con varios valores de M, se ha optado por tener 10 valores, que equivale a la mitad de los pulsos que da una rueda, recordando que su resolución es de 20.\\

 Cuando la placa de Arduino ha registrado los intervalos de tiempo de la rueda en movimiento y después se para, la velocidad se queda con el último tiempo medido y debido al filtro de media móvil, este valor desciende a cero pero con un retraso importante. Para evitar esto se impone una condición. Si durante un intervalo de tiempo no se ha registrado ningún pulso proveniente del encoder, la velocidad se pone a cero. El tiempo que se ha puesto es de 100 ms.




\clearpage\thispagestyle{empty}\cleardoublepage
\subsection{Controlador del Motor DC}
Para que el robot pueda seguir las intrucciones de manera correcta y los algoritmos se puedan efectuar de manera acertada, el robot debe incorporar un control de velocidad lineal,el control se aplica a los dos moteres DC de las ruedas motrices. En el siguiente diagrama se puede ver el esquema del control.\\
\tikzstyle{block} = [%bloque
draw,
minimum width=0.6cm,
minimum height=0.3cm
]

\begin{tikzpicture}

% Sum shape
\node[draw,
circle,
minimum size=0.5cm,
fill=Rhodamine!50
] (sum) at (0,0){};

\draw (sum.north east) -- (sum.south west)
(sum.north west) -- (sum.south east);

\draw (sum.north east) -- (sum.south west)
(sum.north west) -- (sum.south east);

\node[left=-3pt] at (sum.center){\tiny $+$};
\node[below=-3pt] at (sum.center){\tiny $-$};



% Sensor block sampler
\node [block,
fill=SeaGreen, 
right=1cm of sum
]  (sampler) {   \begin{tikzpicture}
	\draw  ++ (0, 0)
	to [nos,] ++ (0.5,0) ;
	\end{tikzpicture}
};

% Controller
\node [block,
fill=Goldenrod,
right=1cm of sampler
]  (controller) {$PID(z)$};

%sum2
\node[draw,
circle,
minimum size=0.5cm,
fill=Rhodamine!50,
right=0.4 cm of controller
]  (sum2) {};

\draw (sum2.north east) -- (sum2.south west)
(sum2.north west) -- (sum2.south east);

\draw (sum2.north east) -- (sum2.south west)
(sum2.north west) -- (sum2.south east);

\node[left=-3pt] at (sum2.center){\tiny $+$};
\node[above=-3pt] at (sum2.center){\tiny $+$};

% FeedForward
\node [block,
fill=OrangeRed,
above left= 0.5cm and 0.25cm of controller
]  (feedforward) {$F$};

% Sensor block sampler
\node [block,
fill=SeaGreen, 
left=1cm of feedforward
]  (sampler2) {   \begin{tikzpicture}
	\draw  ++ (0, 0)
	to [nos,] ++ (0.5,0) ;
	\end{tikzpicture}
};
% Sensor block sampler
\node [block,
fill=SeaGreen, 
right=0.8cm of sum2
]  (ZOH) {$ZOH$};

% Entrada
\node [block,
fill=BlueGreen,
left=1cm of sum
]  (voltage) {$U(s)$};

% System G(s)
\node [block,
fill=SpringGreen, 
right=0.5cm of ZOH
] (system) {$G(s)$};

% Sensor block H(s)
\node [block,
fill=SeaGreen, 
right= 0.5cm of system
]  (sensor) {$H(s)$};


% Arrows with text label
\draw[-stealth] (sum.east) -- (sampler.west)
node[midway,above]{};

\draw[-stealth] (sampler.east) -- (controller.west)
node[midway,above]{};

\draw[-stealth] (controller.east) -- (sum2.west) 
node[midway,above]{};

\draw[-stealth] (ZOH.east) -- (system.west) 
node[midway,above]{};
\draw[-stealth] (system.east) -- (sensor.west) ;
\draw[-stealth] (sensor.east) -- ++ (0.5,0) 
node[midway](output){}node[midway,above]{$y$};

\draw[-stealth] (output.center) |- (0,-3);
\draw[-stealth] (0,-3) |- (sum.south);


\draw[-stealth] (feedforward.east) -| (sum2.north);


\draw[-stealth] (voltage.east) -- (sum.west);


\draw[-stealth] (voltage.east) |- (sampler2.west);
\draw[-stealth] (sampler2.east) |- (feedforward.west);

\draw[-stealth] (sum2.east) |- (ZOH.west);
\end{tikzpicture}


En el diagrama de bloques se pueden ver las siguientes partes que conforman el sistema de control
\begin{itemize}
	\item $U(s)$ corresponde con una entrada escalón de $w$ que es la velocidad en rad/s de la rueda. La entrada se traduce en un valor de PWM que corresponde con un valor de voltaje el cual está limitado a un valor de 6.5 V.
	\item $F$ corresponde con un controlador feedForward, se ha creado un programa en el microcontrolador, el cual introduce una señal de PWM al puente H y a su vez toma las lecturas proporcionadas por el sensor de velocidad y  se obtiene una tabla de entradas PWM que corresponde a un nivel de voltaje y una salida de rad/s que corresponde al giro de la rueda. Se hace para varios valores y a partir de los datos se hace una regresión líneal de la zona mas linealizada de los datos o si se puede hacer sobre todas las medidas mejor. En la Figura ~\ref{fig:feedforward} se puede ver las muestras obtenidas y la ecuación resultante que se aplica en el controlador de cada motor. Para realizar las medidas se ha tenido en cuenta el rozamiento, esto implica que las pruebas se han realizado con el robot montado y sobre la superficie del robotario.
	\begin{figure}[p]
		
		\begin{subfigure}[b]{0.5\linewidth}
			\centering
			\includegraphics[width=1\linewidth]{controlMotor/FeedforwardR1_R}
			\caption{Rueda derecha Robot1}
			\label{fig:feedforwardR1_R}
			\vspace{4ex}
		\end{subfigure}%%	
		\begin{subfigure}[b]{0.5\linewidth}
			\centering
			\includegraphics[width=1\linewidth]{controlMotor/FeedforwardR1_L}
			\caption{Rueda izquierda Robot1}
			\label{fig:feedforwardrR1_left}
			\vspace{4ex}
		\end{subfigure}	
		\begin{subfigure}[b]{0.5\linewidth}
			\includegraphics[width=1\linewidth]{controlMotor/FeedforwardR2_R}
			\caption{Rueda derecha Robot2}
			\label{fig:feedforwardR2_R}
			\vspace{4ex}
		\end{subfigure}
		\begin{subfigure}[b]{0.5\linewidth}
			\includegraphics[width=1\linewidth]{controlMotor/FeedforwardR2_L}
			\caption{Rueda izquierda Robot2}
			\label{fig:feedforwardrR2_left}
			\vspace{4ex}
		\end{subfigure}	
		\begin{subfigure}[b]{0.5\linewidth}
			\includegraphics[width=1\linewidth]{controlMotor/FeedforwardR2_R}
			\caption{Rueda derecha Robot2}
			\label{fig:feedforwardR3_R}
			\vspace{4ex}
		\end{subfigure}
		\begin{subfigure}[b]{0.5\linewidth}
			\includegraphics[width=1\linewidth]{controlMotor/FeedforwardR2_L}
			\caption{Rueda izquierda Robot2}
			\label{fig:feedforwardrR3_left}
			\vspace{4ex}
		\end{subfigure}
		
		\caption{FeedForward}
		\label{fig:feedforward}
	\end{figure}
\newpage
	\item $PID(z)$ corresponde al controlador. El control de acción directa no basta para mantener una velocidad debido a que las condiciones en las que se ha hecho el control de acción directa no son las mismas en todo momento, por ello es necesario un control realimentado, para ello se usa uno de los controladores mas usados y conocidos que es el PID. Se desconoce como es la planta, se tiene un modelo del motor DC pero el modelo no basta para conocer el comportamiento real. Por ello para la sintonización de los parámetros del PID se usa el método de ziegler-Nichols para una primera aproximación y luego se va ajustando hasta conseguir una respuesta deseada. Ademas de la sintonización de los parámetros del controlador, se implementa un modulo conocido como Anti-Wind-up.
	
	 \subitem El Wind-Up es un problema que se tiene con el control integral, este controlador tiene memoria y, debido a que acumula valores anteriores del error esta acción integradora crece incluso cunado se llegue al valor referencia, y cuando se produce un cambio en la referencia el controlador tarda en reaccionar debido a los valores acumulados que guarda, por ello se propone dentro del control PID el denominado Anti-Wind-Up, básicamente lo que hace es que cuando llega a un valor máximo el integrador deja de acumular valores y cuando se cambia el signo del error se borra la memoria del controlador.
	 
 	Implementado en código para introducirlo en el microcontrolador quedaría de la siguiente forma:
 	
	 \begin{lstlisting}
	int pidI(double wI)
	{
		currentTimeI=millis();
		elapsedTimeI=currentTimeI - previousTimeI;
		int outputI=0;
		errorI = setpointWI - wI;   
		double aux;
		//condicion de error minimo
		if(errorI<0){
			aux=-errorI;
		}
		else{
			aux=errorI ;          
		}
		if(aux>=minError){
		
			cumErrorI = cumErrorI + errorI * elapsedTimeI; 
			//se resetea el error acumulativo cuando se cambia de signo
			if(lastErrorI>0 && errorI<0){
				cumErrorI=errorI* elapsedTimeI;
			}
			if(lastErrorI<0 && errorI>0){
				cumErrorI=errorI* elapsedTimeI;
			}
			//se establece un maximo de error
			if(cumErrorI>maxcumError||cumErrorI<-maxcumError)cumErrorI=maxcumError;
			 // calcular la derivada del error
			rateErrorI = (errorI - lastErrorI) /elapsedTimeI;        
			 // calcular la salida del PID 
			outputI = static_cast<int> (round(KI_p*errorI  + KI_i*cumErrorI + KI_d*rateErrorI));    
			//se guarda el error anterior
			lastErrorI = errorI;
		}
		
		else{
			outputI=0;
		}
		return outputI;
	
	}
	 \end{lstlisting}
	 
	Se ha puesto solo el controlador de la rueda izquierda pero el de la derecha es igual, como se puede apreciar en el codigo cada un cierto tiempo que es el denominado elapsedTimeI el controlador actúa haciendo una suma acumulada denominada cumErrorI que equivale al control integral. Se puede apreciar que en el codigo de programación se tienen unos límites denominados minError a partir del cual el controlador actua, este límite esta puesto debido la cuantificación del voltaje en PWM, como solo se puede adquirir un determinado voltaje se establecen un margenes de error aceptables, si no fuera asi el controlador oscilaria tratando de alcanzar la señal de referencia.\\
	El anti WindUp esta en la línea 26 donde se establece un error máximo de suma. Y el reset esta escrito desde la línea 19 hasta la línea 24 
	\item $G(s)$ es el motor DC, en varios libros y documentos existen diversos modelos de un motor DC para el control, en este caso no es necesario modelar el motor, debido a que se va a trabajar sobre él directamente y se sintonizará el control sobre la planta real.
	\item $H(s)$ es el sensor, que se refiere al encoder óptico, dicho encoder nos da pulsos que equivalen a $\theta=\frac{2\pi}{N}$ radianes, si se mide el tiempo entre pulsos se tiene la velocidad angular de la rueda.
\end{itemize}
En conjunto los bloques forman el lazo de control, que mantiene una velocidad en referencia a la entrada.\\
En el diagrama de bloques anterior no se ha tenido en cuenta las perturbaciones que se obtienen, y debido a que el encoder y los motores no son de buena calidad y debido a las vibraciones del robot causadas por el movimiento, se tiene una alteración significativa de la lectura de pulsos. En el siguiente apartado se comenta el tratamiento del ruido y su solución.  

\subsubsection{sintonización de parámetros PID}
Debido al ruido del sistema, se decide por no incluir la parte derivativa del controlador PID, basta con poner $K_{d}=0$, esto se debe a que la acción derivativa amplifica el ruido, se podría poner un filtro a la acción derivativa, pero sería complicar el control de manera innecesaria y como se muestra a continuación con un control PI junto al control de acción directa es mas que suficente.\\
Para la sintonización de los parámetros se recurre al método de Ziglers-Nichols, esto se debe a que se tiene una planta experimental y este método va a ser de ayuda para dar con los parámetros del controlador.\\
Se toma como ejemplo de sintonización el robot1.
Lo primero que se hace es aplicar una ganacia K en el lazo de control, se aumenta dicha ganacia hasta que se llegue a un comportamiento oscilatorio. Con la K que ha causado el comportamiento oscilatorio y el periodo de la oscilación se hallan los prámetros del controlador PI.
\begin{figure}[htbp]

	\begin{subfigure}[b]{0.52\linewidth}
		\centering
	\includegraphics[width=0.95\linewidth]{controlMotor/RespuestaEscalonRD}
	\caption{}
	\label{fig:respuestaescalonrd}
	\end{subfigure}
\quad
	\begin{subfigure}[b]{0.52\linewidth}
		\centering
		\includegraphics[width=1\linewidth]{controlMotor/RespuestaEscalonRI}
		\caption{}
		\label{fig:respuestaescalonri}
	\end{subfigure}
\end{figure}\\

De la representación gráfica se obtiene T, para la rueda derecha $TD=0.798$ y $ Ti=0.864$, con estos valores y con k=5.3 se obtiene la siguiente tabla.\\
\begin{tabular}{|c|c|c|c|c|c|c|}
	\hline
	& $KD_{p}$ & $KD_{i}$ & $kD_{d}$ & $KI_{p}$ & $KI_{i}$ & $kI_{d}$ \\
	\hline
	P & 1.95 &  &  & 1.95 &  &  \\
	\hline
	PI & 1.755 & 0.6384 &  & 1.755 & 0.6912 &  \\
	\hline
	PID & 2.34 & 0.399 & 0.09975 & 2.34 & 0.432 & 0.108 \\
	\hline
\end{tabular}\\

En la Figura ~\ref{fig:PI_Nichols}, se puede apreciar la respuesta de los motores, se puede ver que en el motor derecho se tiene una oscilación y el motor izquierdo llega al asentamiento pero tiene una sobre-elongación grande.


\begin{figure}[h]
	
	\begin{subfigure}[b]{0.52\linewidth}
		\centering
		\includegraphics[width=1\linewidth]{controlMotor/RespuestaEscalonPID_R}
		\caption{PI derecha}
		\label{fig:PI_R}
	\end{subfigure}
	\quad
		\begin{subfigure}[b]{0.52\linewidth}
		\centering
		\includegraphics[width=1\linewidth]{controlMotor/RespuestaEscalonPID_I}
		\caption{PI izquierda}
		\label{fig:PI_I}
	\end{subfigure}
	\caption{PI}
	\label{fig:PI_Nichols}
\end{figure}

 Se ajusta manualmente los parámetros del PI y se llega al resultado mostrado ne la Figura ~\ref{fig:PI_sintonizado}. Esta vez corrige de manera mas eficaz, se tiene una pequeña oscilación que se debe a parámetros mecanicos del robot, a la situación de las ruedas, alineación de los ejes etc y al propio ruido de los encoders. Aún así el control es bastante efectivo.
 \begin{figure}[h]
 	
 	\begin{subfigure}[b]{0.49\linewidth}
 		\centering
 		\includegraphics[width=0.8\linewidth]{controlMotor/PI_sintonizado_R}
 		\caption{PI rueda derecha}
 		\label{fig:PIsintonizado_R}
 	\end{subfigure}
 	\quad
 	\begin{subfigure}[b]{0.49\linewidth}
 		\centering
 		\includegraphics[width=0.8\linewidth]{controlMotor/PI_sintonizado_I}
 		\caption{PI rueda izquierda}
 		\label{fig:PIsintonizado_I}
 	\end{subfigure}
 	\caption{PI}
 	\label{fig:PI_sintonizado}
 \end{figure}

Y la respuesta de los otros dos robots se puede apreciar en la figura [figura]\\

las constantes de control de los 3 robots son los siguientes[tabla]\\

 
\subsection{Navegación del robot}

\begin{figure}[htbp]
	
	\begin{subfigure}[b]{0.52\linewidth}
		\centering
		\includegraphics[width=0.7\linewidth]{navegacion/RuedaRobotDibujo}
		\caption{Rueda de robot}
		\label{fig:ruedarobotdibujo}
	\end{subfigure}
	\quad
	\begin{subfigure}[b]{0.52\linewidth}
		\centering
		\includegraphics[width=0.7\linewidth]{navegacion/ArcoRecorridoRobot}
		\caption{Arco que recorre robot}
		\label{fig:arcorecorridorobot}
	\end{subfigure}
\end{figure}
Una vez que se ha sintonizado el controlador y que se obtiene un movimiento suave y uniforme, se calculan las ecuaciones de cinemáticas que gobiernan al robot y sirven para establecer la navegación del robot.
Se puede ver en la Figura ~\ref{fig:arcorecorridorobot} la distancia entre las ruedas del robot(L) y el arco que forma el robot al moverse,$S1$ y $S2$.
La rueda izquierda forma un arco de radio x y la rueda derecha un arco de radio $x+l$
	siendo las longitudes de ambos arcos:
	
	\begin{center}

		$S2=x*\Delta\varphi=R\Delta\theta$\\
	$S1=(x+L)*\Delta\varphi=R\Delta\theta$\\
	\end{center}
	Donde  $R\Delta\theta$ es la distancia que avanza el robot y $\Delta\theta=\frac{2\pi}{N}i$ son los pasos de encoder del robot, siendo R el radio de la rueda e i los pulsos de encoder. Se hace la resta de S1 y S2.
	\begin{center}
	 $S1-S2= L\Delta\varphi = R*\Delta\theta_{d}-R\Delta\theta_{i}$\\
	 $\Delta\varphi = \frac{R(\Delta\theta_{d} - \Delta\theta_{i})}{L}$\\
	\end{center}
	 Se deriva respecto al tiempo.\\
	 \begin{center}
	 $ \frac{\Delta\varphi}{\Delta t}=  \frac{R(\Delta\theta_{d} - \Delta\theta_{i})}{L*\Delta t} = \frac{r(w_{d}-w{i})}{L}$\\
	
	\end{center}
 de esta manera se tiene la velocidad angular del robot que depende de la velocidad angular de cada rueda y de la distancia que las separa. Y por último la velocidad lineal del robot viene dada por la siguiente expresión.
 
 \begin{center}
 	$V= \frac{v_{d} + v_{i}}{2} = \frac{R(w_{d}-w{i})}{2}$
 \end{center}
Con las expresiones obtenidas, lo siguiente que se hace es ponerlas en forma matricial para poder pocersarlas en el robot.
\[
\begin{pmatrix}
W\\
V
\end{pmatrix}
= \begin{pmatrix}
\frac{R}{L} & \frac{-R}{L}\\
\frac{R}{2} & \frac{R}{2}
\end{pmatrix}*
\begin{pmatrix}
w_{D}\\
w_{I}
\end{pmatrix}
\]
Las consignas de orden para el robot son $W$ y $V$ y el robot tendra como entrada $w_{D}$ y $w_{I}$, se resuelve la ecuación matricial y se obtiene las velocidades angulares de las ruedas en función de $W$ y $V$.
\[
\begin{pmatrix}
w_{D}\\
w_{I}
\end{pmatrix}
= \begin{pmatrix}
\frac{L}{2R} & \frac{1}{R}\\
\frac{-L}{2R} & \frac{1}{R}
\end{pmatrix}*
\begin{pmatrix}
W\\
V
\end{pmatrix}
\]

Esta matriz se puede implementar en el código del robot y se tiene el gobierno del movimiento del robot.


\section{Localización de los robots}\label{ch:localizacionRobots}
\begin{figure}[h]
	\centering
	\includegraphics[width=0.5\linewidth]{Localizacion/EjemploMarker}
	\caption{Robots Con Markers para identificación y localización}
	\label{fig:ejemploMarkerrobot}
\end{figure}

Para localizar los distintos robots involucrados en el robotario se usan cámaras, se tiene una cámara cenital y cada robot dispone de una cámara conectada a una raspberryPi, que son las cámara mencionadas en el capítulo ~\ref{ch:HardwareYsoftware} . Para obtener información del entorno con las cámaras se usa la librería OpenCV, la librería tiene múltiples aplicaciones para el reconocimiento del entorno, en particular, para determinar la posición y orientación de los robots se van a usar unas figuras denominadas Markers que constan de un fondo negro, y en blanco tienen un identificador, la información que se puede almacenar en el Marker consta de 16 bits. El borde es negro lo que permite su rápida localización en el entorno, además la formar del Marker es cuadrada, con todos los lados iguales. Cada esquina del Marker se identifica de manera única y se guarda su localización en la imagen, en pixeles, esto permite obtener la orientación del Marker en el mundo observado por la cámara, se detalla el proceso más adelante. Para identificar y crear los diferentes Marker se debe usar un diccionario prefijado por la librearía ArUcO mencionada en el capitulo ~\ref{ch:HardwareYsoftware} (OpenCV), en los cuales está la codificación de los diferentes Markers, tienen 16 diccionarios cada cual está codificado de manera diferente, las principales características del diccionario son las dimensiones del Marker y el número de Markers a usar.\\

Para generar un Marker se usa un programa que proporciona OpenCV donde se puede generar un solo Marker o varios a la vez, se guardan en formato .PNG y luego se pueden imprimir, se debe elegir el tamaño que se quiere que tenga el Marker, el diccionario a usar y el identificador, y como resultado se obtienen los diferentes Marker para los robots, en la Figura ~\ref{fig:ejemploMarkerrobot} se muestra un ejemplo de como son los Markers y la colocación en los robots.Se puede apreciar como cada robot dispone de un Marker en la parte delantera del robot y otro en la parte superior, esto es así para permitir la localización entre ellos y para permitir la localización en el robotario y tener un sistema de localización global.\\
	\begin{figure}
	\centering
	\includegraphics[width=0.5\linewidth]{Localizacion/board}
	\caption{Markers para calibración}
	\label{fig:eMarkerBoard}
\end{figure}
\subsection{Calibración de cámaras}
Para poder localizar los robots, primero se deben calibrar las cámaras, existen diversas maneras de calibrar las cámaras. Se ha seguido la proporcionada por ArUcO, la cual consiste en disponer de una tabla de Markers, como la mostrada en la Figura~\ref{fig:eMarkerBoard} donde se debe medir las dimensiones físicas de los Markers, que deben ser todos del mismo tamaño y cada uno con un identificador diferente, también se debe medir la distancia que separa los distintos Markers, que también debe ser igual para todos. Con el programa proporcionado por Aruco,se pasan los parámetros físicos de los Markers a uno función de la librería de ArUcO y se obtienen los coeficientes de distorsión de la cámara y  los parámetros de la longitud focal y los centros ópticos ,correspondientes con los ejes x e y del mundo observado con la cámara, con estos últimos se crea una matriz única para cada cámara.
\begin{equation}
\begin{bmatrix}
fx & 0 & cx\\
0 & fy & cy\\
0 & 0 & 1
\end{bmatrix}
\end{equation} 

La matriz y los coeficientes de distorsión se utilizan para localizar en unidades físicas (metros) donde se encuentran los diferentes Markers identificados por la cámara. Tambiénse puede conocer la orientación respecto a los ejes de la cámara. Para ello se emplean las esquinas de cada Marker, siendo identificada cada esquina y guardada. Se puede identificar la orientación calculando el centroide del cuadrado y el ángulo respecto a una esquina de referencia, esta esquina es siempre la superior izquierda del Marker. Observando la Figura \ref{fig:eMarkerBoard} las esquinas se cuentan en el sentido de las agujas del reloj empezando por la esquina superior izquierda y acabando por la inferior izquierda.\\
\subsection{Identificación y localización}
La cámara tiene sus  propios ejes, siendo el centro del foco el origen de coordenadas , la cámara tiene el eje Z apuntado fuera de su foco, el eje X apunta desde la izquierda hasta la derecha de la imagen y el eje Y apunta desde arriba hasta abajo de la  imagen. Es importante tener en cuenta la orientación de los ejes de la cámara para poder localizar los robots en el mundo.
Para localizar un Marker y estimar su posición en coordenadas XYZ en metros, OpenCV tiene una serie de funciones que permiten hacerlo, primero se debe identificar donde se encuentra el Marker en la imagen que proporciona la cámara, esto se hace con $ cv::aruco::detectMarkers(image, dictionary, corners, ids);$ , esta función te devuelve un array de vectores con la localización de las esquinas de cada Marker en pixeles y con el identificador del marker localizado, los parámetros que se tienen que pasar son el frame o captura de la imagen y el diccionario que se está usando.\\
El siguiente paso es estimar la posición, para ello se necesitan los coeficientes de distorsión, la matriz de la cámara y la dimensión del Marker en metros, siendo estos cuadrados basta con conocer un lado. la función que permite conocer la posición y orientación es la siguiente:
\begin{lstlisting} 
cv::aruco::estimatePoseSingleMarkers(corners, marker_length_m,camera_matrix, dist_coeffs, rvecs, tvecs); 
\end{lstlisting}
A esta función se le deben pasar los parámetros mencionados y te devuelve un vector de traslación $tvecs$ y un vector de rotación $rvecs$ con estos vectores se puede conocer la posición de cada marker y su respectiva orientación además de que permite conocer el sistema de referencia del Marker, siendo su origen de coordenadas el centro del Marker.\\

En la Figura ~\ref{fig:EjesCoordenadas} se puede ver como se localizan los Markers y los ejes de la cámara y el Marker, siendo para ambos el verde el eje X el rojo el eje Y y el azul el eje Z, 

Con el vector de rotación no bastaría para conocer la orientación, es preciso transformarlo en una matriz de rotación y esto se consigue con el algoritmo de Rodrigues \cite{rodrigues}.
\begin{figure}[h]
	
	\begin{subfigure}[b]{0.62\linewidth}
		\centering
		\includegraphics[width=0.9\linewidth]{PoseEstimation}
		\caption{Localizacion de Marker}
		\label{fig:poseEstimation}
	\end{subfigure}
	\quad
	\begin{subfigure}[b]{0.52\linewidth}
		\centering
		\includegraphics[width=0.5\linewidth]{ejesCamara}
		\caption{Ejes de la cámara}
		\label{fig:ejescamara}
	\end{subfigure}
	\caption{Ejes de coordenadas}
	\label{fig:EjesCoordenadas}
\end{figure}
En la Figura ~\ref{fig:EjesCoordenadas} se aprecia como la cámara tiene su sistema de coordenadas y el marker tiene otro, por ello es necesario el vector de rotación y el vector de traslación, que permite posteriormente dar instrucciones a los robots para moverse en el robotario o localizarse unos a otros.
Ademas de lograr obtener la posición de los robots en xyz, en metros, respecto al sistema de referencia de la cámara, se debe obtener su orientación respecto al mismo. Para ello se emplea la matriz de rotación calculada a partir del vector de rotación proporcionado por la función descrita anteriormente. Con esta matriz se puede calcular el heading, que se corresponde con los grados girados respecto al eje Z de la cámara. La matriz de rotación es la matriz resultante de haber realizado las rotaciones respecto al eje x, y ,z, para trasladar el sistema de referencia de la cámara al sistema de referencia del robot. Las matrices de rotación respecto a los ejes son:
\begin{equation}
Rx(u)=
\begin{bmatrix}
1 & 0 & 0\\
0 & cos(u) & -sin(u)\\
0 & sin(u) & cos(u)
\end{bmatrix}    
\end{equation}

\begin{equation}
Ry(v)=
\begin{bmatrix}
cos(v) & 0 & sin(v)\\
0 & 1 & 0\\
-sin(v) & 0 & cos(v)
\end{bmatrix}  
\end{equation}


\begin{equation}
Rz(w)=
\begin{bmatrix}
cos(w) & -sin(w) & 0\\
sin(w) & cos(w) & 0\\
0 & 0 & 1
\end{bmatrix}  
\end{equation}

Como resultado la multiplicación de las matrices da la matriz de rotación.

$R=R_{z}(w)R_{y}(v)R_{z}(u)=$
\begin{equation}
\begin{pmatrix}
cos(w)cos(v) & sin(u)sin(v)cos(w)-cos(u)sin(w) & cos(u)sin(v)cos(w)+sin(u)sin(w)\\
cos(v)sin(w) & sin(u)sin(v)sin(w)+ cos(u)cos(w) & cos(u)sin(v)sin(w)-sin(u)cos(w)\\
-sin(v) & sin(u)cos(v) & cos(u)cos(v)
\end{pmatrix}
\end{equation}

De la matriz de rotación se puede despejar el ángulo de rotación respecto al eje Z, a partir de los elementos de la matriz $R_{11}$ y $R_{21}$, se dividen ambos terminos y se despeja el angulo $w$ y de esta manera se obtiene el heading.
\begin{equation}
\frac{R_{21}}{R_{11}}=\tan(w)
\end{equation}
Con este parámetro ya se obtiene la orientación del robot respecto al eje Z del sistema de referencia de la cámara.

\subsection{Ruido en la estimación de posición}\label{ch:RuidoPosicion}
Se ha detectado que en la localización existe una variación en la posición. Las coordenadas devueltas por el programa oscilan incluso cuando el objeto está parado.
\section{Diseño y configuración de la red de Comunicación}\label{ch:RedLan}
En este capítulo se detalla cómo es la estructura de comunicación del Robotario que permite dar instrucciones a los robots usando la localización y la navegación. La comunicación de los robots se lleva a cabo de manera inalámbrica. Para ello se ha elegido establecer la comunicación con los distintos robots mediante WiFi. Debido a que las placas de Arduino tienen una antena incorporada para la conexión a redes inalámbricas con la tecnología WiFI, es la opción más económica sin necesidad de añadir un módulo extra. Además, como el espacio donde se realizan los experimentos es limitado y de pequeña extensión está asegurada la cobertura.\\
 Para poner en situación el alcance de la comunicación, se hace un cálculo aproximado. Se tiene que la antena de la placa de Arduino tiene una PIRE de 18 dBm. La ganancia de la antena del router se desconoce, pero consultando el mercado se sabe que están en un valor desde 2 dBi a 9 dBi. Como se quiere probar el caso peor se toma 2 dBi, la sensibilidad del router es de 80 dBm y como frecuencia de referencia se toma 2.4GHz que es la correspondiente a la tecnología WiFi. Con estos datos y a partir de la ecuación de Friis\cite{cardama}
\begin{equation}
	Pr(dBm)=PIRE(dBm)+ Gr(dBi) + 20·log(\frac{\lambda}{4\pi d})
\end{equation} 
siendo Pr, la potencia recibida en bornes de la antena del receptor.
Despejando se obtiene $d$ que es la distancia máxima y da como resultado 994 metros. Hay que señalar que no se han considerado obstáculos, paredes, etc. Pero sirve para estimar aproximadamente el alcance. La localización actual del router permite una visión de los robots sin obstáculos por medio. Para el caso contrario, del router a la placa de Arduino, se tiene que la sensibilidad de Arduino es de 96 dBm, mucho mayor que la del router, por lo tanto, el caso anterior es el más restrictivo.\\
Las dimensiones del Robotario son adecuadas para esta comunicación, se recuerda que son 180cm de largo y 132 cm de ancho. El router se encuentra como máximo a tres metros de cualquier punto del Robotario.

\subsection{Red LAN}
\begin{figure}[h]
	\centering
	\includegraphics[width=0.7\textwidth]{CreacionInfraestructura/RedRobotario}
	\caption{Red LAN Robotario }
	\label{fig:ConfiguracionRed}
\end{figure}
Se tiene una red LAN con topología de red en estrella. En la figura~\ref{fig:ConfiguracionRed} se puede ver una representación gráfica de las conexiones, donde los robots se conectan de forma inalámbrica y el servidor está conectado mediante una conexión cableada, si bien puede ser también inalámbrica. En la situación actual del Robotario, se tienen 3 robots y el servidor\\
La transmisión de la información se realiza a través del protocolo UDP. Se usa este protocolo y no otro como TCP, para evitar retrasos en el envío de información. Un retraso puede provocar que, al dar una instrucción al robot esta no llegue a tiempo y se produzca un fallo en el algoritmo. Además, UDP es un protocolo más ligero al no mantener la conexión. Un factor a tener en cuenta es que los paquetes no van a llegar en orden o la información se  puede perder, pues no se garantiza la entrega de información con el protocolo UDP. Una solución ante este problema es enviar muchos paquetes en un intervalo pequeño de tiempo, teniendo cuidado con no congestionar la red. Esto puede ayudar en la comunicación servidor-robot. Como el servidor actualmente envía instrucciones al robot relacionadas con la posición, si el intervalo de envío es pequeño un paquete que llegue en desorden o se pierda, no afecta en gran medida, pues si el intervalo es más pequeño que la distancia avanzada el error es insignificante. En los datos que se requieran del robot, tales como velocidad, u otro sensor a bordo, basta con poner una marca de tiempo junto al envío de la información y ordenar los paquetes según van llegando al servidor.


La información que se envía mediante la red LAN tiene una estructura definida para que los robots o el servidor puedan identificar la información que les llega, de dónde les llega y cómo procesarla. Los paquetes tienen la siguiente cabecera:
\begin{itemize}
	\item \textbf{Id} identificador del origen
	\item \textbf{Op}. Código de operación, está en hexadecimal.
	\item \textbf{Len}. Longitud de los datos que se envían.
\end{itemize}
Cada robot dispone de un identificador que es un numero en decimal, el servidor también dispone de un identificador siendo este el 0. Las operaciones que se quieren ejecutar están almacenadas en una librería y codificadas en hexadecimal. Todos los robots y el servidor deben usar la misma librería de instrucciones y, por último, el apartado de la longitud sirve para detectar errores. Si llega un paquete con un  tamaño de información distinto al tamaño de la cabecera o a la longitud de la información que viene determinado por len y corresponde a datos enviados, el paquete recibido se desecha. Se muestra la estructura del mensaje enviado/recibido en la figura ~\ref{fig:estructuraMensaje} que comparten los robots y el servidor para poder comunicarse entre ellos. También se tienen definidas las instrucciones que se pueden ejecutar en el Robotario.\\
\begin{figure}[!h]
	 \begin{lstlisting}
		const int MAXDATASIZE =255; //numero de bytes que se pueden recibir
		const int HEADER_LEN = sizeof(unsigned short)*3;
		struct appdata{
			
			unsigned short id; //identificador
			unsigned short op; //codigo de operacion
			unsigned short len; /* longitud de datos */
			unsigned char data [MAXDATASIZE-HEADER_LEN];//datos		
		};
		//operacion error
		#define OP_ERROR            0xFFFF //Error de operacion
		#define OP_SALUDO           0x0001//Se verifica que el robot esta encendido y conectado a la red
		#define OP_MOVE_WHEEL       0x0002//da orden para mover ruedas wd,wi
		#define OP_STOP_WHEEL       0x0003//para las ruedas wd,wi
		#define OP_VEL_ROBOT        0X0004//devuelve la velocidad de las ruedas en rad/s wd,wi
		#define OP_IMU              0x0005//devuelve lectura de giroscopo y acelerometro
		#define OP_STOP_SERIAL      0X0006//para la comuniacion serie del robot
		#define OP_POSITION         0x0007//manda la posicion inicial de robot
		//broadcast
		#define OP_BROADCAST        0x9999//operacion de difusion

 \end{lstlisting}
 	\caption{Estructura de mensaje enviado/recibido y operaciones de instrucción }
 	\label{fig:estructuraMensaje}
 \end{figure}
La estructura appdata que aparece en la figura~\ref{fig:estructuraMensaje} corresponde a la de los paquetes que se acaba de comentar, donde se tiene un array para poder enviar datos, como velocidad  del robot, información del acelerómetro etc. Después de la estructura se pueden apreciar las instrucciones que se puede ejecutar. Las instrucciones se pueden ir ampliando y aumentar la complejidad del Robotario. El tamaño máximo de datos enviados consta de 255 bytes. Es un tamaño pequeño lo que dificulta la fragmentación del paquete debido a la unidad de transmisión máxima (MTU) de una red, que es el tamaño máximo permitido en bytes de un paquete que se puede transmitir en una red. Si se supera el tamaño máximo el paquete se fragmenta. La red actual tiene una MTU de 1500 bytes.
\newpage
\subsection{Calidad de servicio}
Se hace un análisis de los parámetros de la red para poder estimar sus límites y hacer una aproximación de la cantidad de dispositivos que se pueden conectar. 
\subsubsection{Ancho de banda}
 Se cuenta con un ancho de banda en la comunicación inalámbrica de 1750 Mbps, siendo esta la capacidad máxima de datos que se pueden transmitir a través del canal. Normalmente los valores reales son inferiores. Aun así, un valor algo inferior no es un problema para la comunicación del Robotario, pues se envían paquetes con una longitud máxima de 255 bytes más la cabecera UDP que son 8 Bytes y las demás cabeceras de encapsulación se tiene un tamaño máximo de 297 bytes. Comparando la longitud máxima de los paquetes con el ancho de banda, se observa que este es suficiente para transmitir la información con los 3 robots actuales, y se podria soportar más dispositivos en la red transmitiendo información.
\subsubsection{Latencia o retardo}
  La latencia se define como el tiempo que  tardan los flujos de datos en llegar a su destino. El retardo depende del tiempo de transmisión que se define como:
  \begin{equation}
 t_{t}=\frac{L}{C}
  \end{equation} 
  Donde L es la longitud máxima del paquete (297 bytes) y C se refiere al ancho de banda (1750Mbps), calculando se obtiene un tiempo de transmisión $t_{t}=1.36\mu s$. 
  El retardo debido a la congestión depende solo del router, y se produce por un tráfico alto de paquetes en el router o por como procesa éste la información.\\
\begin{figure}[!ht]
	\centering
	\includegraphics[width=0.5\linewidth]{CreacionInfraestructura/latencia}
	\caption{Captura de pantalla de una prueba de latencia de la red}
	\label{fig:latencia}
\end{figure}
En la figura~\ref{fig:latencia} se tiene la captura de una prueba de latencia de la red, esta prueba se ha realizado con 3 robots transmitiendo información al servidor y viceversa. Se ha realizado entre el servidor que está conectado por cable y una raspberryPi conectada a la red mediante WiFi. La prueba sirve para estimar la latencia. Se han hecho diversas pruebas, para encontrar el peor caso y como máximo se obtiene el resultado mostrado que es de 119 ms de retardo. Este tiempo máximo puede afectar de manera significativa en la transmisión de ordenes al robot, pero en general la latencia está por debajo de los 10ms que es un parámetro muy bueno y permite una transmisión de información de manera fluida.\\
\subsubsection{Jitter y perdida de paquetes}
\begin{figure}[!h]

	\centering
	\includegraphics[width=0.7\linewidth]{CreacionInfraestructura/jitterS}
	\caption{Analisis de la red con iperf3}
	\label{fig:jitters}

\end{figure}
Se realiza una prueba con iperf3 un programa que permite analizar las redes creando un servidor y cliente y enviando paquetes de longitud a elegir. Se puede elegir la velocidad de transmisión. En la prueba realizada que corresponde a la figura~\ref{fig:jitters} se ha elegido una velocidad de transmisión de 1 Mbps. La prueba se realiza con los robots y el servidor mandado paquetes a la red. En la figura~\ref{fig:jitters} se tiene un error en el parámetro de perdida de paquetes en el primer intervalo, esto se debe a que los paquetes han llegado desordenados y el programa ha contado mal el número de paquetes, es un problema reconocido de la aplicación. Después de este intervalo se corrige el error anterior y se tiene una pérdida de paquetes del 0\%. Este dato es muy bueno pues a pesar de estar enviando paquetes a la red procedentes de otros dispositivos no se ha perdido información esto significa que el router soporta el flujo de datos de manera adecuada.
El último parámetro a medir de la red es el jitter. Se trata de la variabilidad del retardo, se produce debido a la congestión de la red o a cambios en la ruta de los paquetes.
En la figura~\ref{fig:jitters} se puede apreciar que el máximo jitter ocurre en el primer intervalo de tiempo que es el intervalo erróneo. Después de este primer intervalo se puede ver como el jitter no pasa de los 2 ms, lo cual es un buen resultado.\\


Analizando  los parámetros de la red se tiene un buen resultado para los dispositivos conectados, de hecho, se podría conectar más sin problema. El factor que daría problemas dependiendo del número de dispositivos enviando información es la latencia, debido a como el router encamina y procesa la información y a que éste tiene una memoria limitada para almacenar los paquetes que le van llegando hasta que estos puedan ser enviados. Si se tienen muchos dispositivos enviando información en la red se pueden producir retardos debido a la memoria limitada del router.


\section{Demostración del robotario}
Se hacen dos experimentos para comprobar que el robotario funciona. Se quiere demostrar que se puede controlar los robots, se pueden establecer una comunicaciones entre los dispositivos del robotario, y es útil como entorno de pruebas para algoritmos multiagente. Con estos experimentos se pretende conocer los límites tecnológicos del robotario y características como duración de la batería, efectividad de los movimientos y capacidad de comunicación.\\
El primer experimento que se realiza es una reunión de los robots en un punto del robotario. Para ello el punto de reunión será un marker con un identificador que no corresponde a ningún robot. De esta manera se puede apreciar que los robots llegan al punto establecido. El segundo experimento que se realiza es la aplicación de un algoritmo que está en desarrollo. El algoritmo trata de estimar la posición de los robots vecinos, conociendo solo los ángulos internos que forman los robots y las velocidades relativas entre ellos.

\subsection{Experimento I}
El experimento que se realiza es para comprobar el correcto funcionamiento de la localización global proporcionada por la cámara cenital y la comunicación del servidor con los robots involucrados en el experimento. Para ello se ha desarrollado un algoritmo que hace que los robots tengan que ir a un punto en el robotario. Con este algoritmo se demuestra que se pueden orientar y guiar los robot, además de que se puede tener datos de los distintos robots, en este caso es la posición.\\
\begin{figure}[!h]
\begin{subfigure}[b]{0.55\linewidth}
	\centering
	\includegraphics[width=0.7\linewidth]{Demostracion/posicionInicial}
	\caption{Posición inicial experimento I}
	\label{fig:posicioninicial}
\end{subfigure}
\quad
\begin{subfigure}[b]{0.55\linewidth}
	\centering
	\includegraphics[width=0.7\linewidth]{Demostracion/posicionFinal}
	\caption{Posición final experimento I}
	\label{fig:posicionfinal}
\end{subfigure}
\caption{Experimento I}
\label{fig:experimentoI}
\end{figure}
El algoritmo lo que hace es orientar los robots al punto deseado de manera aproximada, y una vez están orientados se desplazan al punto corrigiendo el rumbo mediante las instrucciones proporcionadas por el servidor, de esta manera convergen todos los robots en el punto deseado. En la Figura ~\ref{fig:experimentoI}, se tiene el inicio y el final del experimento. En la Figura ~\ref{fig:posicioninicial} se puede apreciar como los robots empiezan en una posición determinada y mirando cada uno a un sitio distinto. Se recuerda que la parte delantera del robot esta marcada con el vector de rojo que corresponde con el eje y del sistema de referencia del robot y el verde corresponde con el eje x. En la Figura~\ref{fig:posicionfinal} se tiene el final del experimento donde todos los robots han convergido al punto deseado.\\
\begin{figure}[!h]
	\centering
	\includegraphics[width=0.7\linewidth]{Demostracion/calculoVector}
	\caption{}
	\label{fig:calculovector}
\end{figure}

La manera en la cual se orientan, es restando los vectores del robot y el punto deseado, se calcula el angulo del vecto resultante de la resta y esté se resta con el ángulo que tiene el robot que viene dado por la orientación, de esta manera se sabe cuanto tiene que girar el robot. En la Figura~\ref{fig:calculovector} se tiene el calculo vectorial que se hace al robot2 en este experimento. El cálculo se realiza al inicio del algoritmo. Siendo $V_{R}=(x1,y1)$ el vector del robot y $V_{d}=(x2,y2)$ el vector del punto al cual se quiere ir. De la resta se obtiene un nuevo vector, de este se calcula el ángulo respecto al eje X de la cámara. Una vez se tiene el ángulo del vector resultante de la resta,  se resta con el ángulo que posee el robot respecto al eje X que viene dado por la orientación del robot. De esta manera se conoce el error de orientación del robot. Para corregir el error se multiplica por una ganancia y se hace girar el robot hasta que esté aproximadamente orientado al punto deseado, cuando esto ocurra el robot iniciara el rumbo al punto deseado y corregirá la orientación si fuese necesario mediante las consignas proporcionadas por el servidor.
\begin{figure}[!h]
	\centering
	\includegraphics[width=0.8\linewidth]{Demostracion/navegacion}
	\caption{Navegación de  Robots}
	\label{fig:navegacionExperimento1}
\end{figure}
En la Figura ~\ref{fig:navegacionExperimento1} se tiene la navegación que han realizado los robots para llegar al Marker deseado. El punto inicial viene denotado por un cuadrado. Se tiene observando la Figura~\ref{fig:navegacionExperimento1} que el robot3 (azul), ha recorrido mas trayectoria que el robot1 (verde) o el robot2 (rojo). Como el algoritmo es sencillo y solo orienta y corrige el rumbo multiplicando el error por una ganacia determinada, el control es tosco, pero los robots consiguen llegar al destino. En la Figura~\ref{fig:navegacionExperimento1} se tienen lineas continuas, estas significan que se ha perdido la posición del robot. Esto se debe a que cuando los robots están en movimiento la cámara no es capaz de registrar la posición de los robots en cada fotograma. se aprecia que el robot dos ha dado más problemas en el registro de la posición que los demás robots. A pesar de esto no supone un problema para el algoritmo, pues los robots consiguen llegar a la posición final.

\newpage




\subsection{Experimento I}\label{ch:experimento1}
\begin{figure}
	\centering
	\includegraphics[width=0.7\linewidth]{LocalizacionRobots}
	\caption{}
	\label{fig:localizacionrobots}
\end{figure}
\clearpage\thispagestyle{empty}\cleardoublepage
\section{Conclusiones}
Como resultado de las pruebas realizadas se pude concluir que se ha conseguido desarrollar un entorno de pruebas en el cual se puede localizar los robots, establecer una comunicación con ellos y comandar los robots en el robotario. \\

El robot construido es capaz de seguir las ordenes de manera correcta. Los robots tienen una limitación en la velocidad al no poder iniciar con una velocidad menor que $7.5 rad/s$. Se han encontrado limites en la estimación de la velocidad de los robots, lo que limita el uso de algoritmos como se ha visto en el experimento I. Esto último implica que no todos los algoritmos se pueden implementar en robotario, sobre todo los que requieran lecturas de velocidad muy precisas. El control del robot ha resultado ser favorable, pero se tiene incertidumbre en la medida debido al hardware disponible que limita la precisión con la cual los robots se mueven.\\

La comunicación implementada mediante la tecnología WiFi resulta ser suficiente para los robots involucrados, es cierto que se tiene un retraso, y una pérdida de paquetes. El retraso es un factor que se debe aceptar debido al hardware utilizado para la comunicación y la perdida de paquetes no suponen un problema pues se envían muchos paquetes en un intervalo pequeño de tiempo.\\

La posición se ha conseguido estimar con gran precisión, con una variación en la medida como se ha visto en el capítulo ~\ref{ch:RuidoPosicion}, pero respecto al hardware empleado el resultado es favorable.\\

Con los dos algoritmos implementados, se ha podido conocer las limitaciones del hardware. Además al poder mandar instrucciones y localizar los robots de manera correcta, se tiene un entorno de pruebas capaz de implementar algoritmos que requieran estos requisitos.

\subsection{Futuros trabajos}
Como una futura implementación, se comentó que la placa de Arduino tiene una IMU, con un acelerómetro y un giroscopo, este módulo es interesante para tener una precisión mayor cuando el robot realiza giros ya que puede ayudar al controlador PID y por lo tanto al gobierno del motor.\\
También es interesante aplicar un entorno que permita a la persona que quiere llevar su algoritmo una fácil aplicación del mismo. Hasta ahora la manera de implementar un algoritmo se hace a bajo nivel mandando instrucciones de la velocidad a los robots.
\clearpage\thispagestyle{empty}\cleardoublepage




\bibliographystyle{plain}

\bibliography{bibliografia/ref}





\clearpage\thispagestyle{empty}\cleardoublepage
	Implementado en código para introducirlo en el microcontrolador quedaría de la siguiente forma:
 	
	 \begin{lstlisting}
	int pidI(double wI)
	{
 		 currentTimeI=millis();
		elapsedTimeI=currentTimeI - previousTimeI;
		int outputI=0;
		errorI = setpointWI - wI;   
		double aux;
		if(errorI<0){
			aux=-errorI;
		}
		else{
			aux=errorI ;          
		}
		if(aux>=0.30){
		
			cumErrorI += errorI * elapsedTimeI; 
			//se resetea el error acumulativo cuando se cambia de signo
			if(lastErrorI>0 && errorI<0){
				cumErrorI=errorI* elapsedTimeI/1000;
			}
			if(lastErrorI<0 && errorI>0){
				cumErrorI=errorI* elapsedTimeI/1000;
			}
			if(cumErrorI>0){
				if(cumErrorI>MAXCUMERROR)
					cumErrorI=MAXCUMERROR;
				}
				else if(cumErrorI<0){
					if(cumErrorI<(-MAXCUMERROR)){
					cumErrorI=-MAXCUMERROR;
				}
			}
		
			rateErrorI = (errorI - lastErrorI) /elapsedTimeI;         // calcular la derivada del error
			
			outputI = static_cast<int> (round(KI_p*errorI  + KI_i*cumErrorI + KI_d*rateErrorI));     // calcular la salida del PID Kp*errorI  + Ki*cumErrorI + Kd*rateErrorI
			lastErrorI = errorI;
			}
			previousTimeI=currentTimeI;
			return outputI;
		}
		
		else{
			outputI=0;
		}
		return outputI;
	
	}
	 \end{lstlisting}
	 
	Se ha puesto solo el controlador de la rueda izquierda pero el de la derecha es igual, como se puede apreciar en el codigo cada un cierto tiempo que es el denominado elapsedTimeI el controlador actúa haciendo una suma acumulada denominada cumErrorI que equivale al control integral. Se puede apreciar que en el codigo de programación se tienen unos límites denominados minError a partir del cual el controlador actua, este límite esta puesto debido la cuantificación del voltaje en PWM, como se tienen valores discretos de voltaje, se establecen un margen de error aceptable, si no fuera asi el controlador oscilaria tratando de alcanzar la señal de referencia y a la cual no siempre puede llegar.\\
	El Anti wind-up se implementa junto con un borrado de la memoria del integrador, y cuando el error cambia de signo respecto al valor anterior, esto se aprecia en la linea , el proceso que se ha hecho para evitar el wind-up es dejar de sumar cuando se llega a un valor máximo, esto se puede ver de la línea 24 a 32.
\clearpage\thispagestyle{empty}\cleardoublepage

\end{document}