\section{Metodología}\label{ch:metodologia}
Se explica la metodología aplicada para poder llevar a cabo del diseño y construcción del Robotario.
\subsection{Estudio preliminar de entornos de pruebas existentes}
Se realiza un estudio de los proyectos del Robotarium y Duckietown, que han sido referencia principal de este proyecto. Se estudia los componentes dedicados al Robotarium y como son los robots que emplean. Se recopila su experiencia en la construcción de los robots y cuáles han sido los requisitos. Después de haber estudiado con detenimiento estos ejemplos, se plantean los recursos necesarios para construir un entorno de pruebas equivalente.

\subsection{Diseño y construcción de prototipo de robot móvil}
En base a la información recopilada de los ejemplos estudiados, se plantea el robot móvil a construir.\\
Se eligen los componentes necesarios para llevarlo a cabo. Estos deberán ser de fácil acceso y comunes para su fácil sustitución o reparación. Esto conlleva a que sean de bajo presupuesto. Los criterios que se siguen en la elección del hardware son los siguientes:
\begin{itemize}
	\item El tamaño del robot se debe ajustar a las dimensiones del Robotario. 
	\item Se debe seleccionar un sensor para poder estimar la velocidad del robot.
	\item Se debe elegir un microcontrolador para las tareas de bajo nivel del robot.
	\item Se elige una cámara a bordo del robot y un ordenador a bordo que procese la información del entorno y ejecute las tareas de alto nivel requeridas.
	\item El robot debe ser capaz de comunicarse con los distintos dispositivos del Robotario.
	\item El lenguaje de programación debe ser multiplataforma y bien documentado. Compatible con el hardware adquirido
\end{itemize} 
Una vez adquiridos los componentes se realizan las siguientes tareas:
\begin{enumerate}
	\item Comprobar que los sensores del robot estiman de manera adecuada la velocidad. Si no es así se debe tratar este problema.
	\item Verificar el funcionamiento de los elementos que gobernarán la velocidad de los motores y a su vez comprobar el funcionamiento de los mismos.
	\item Montar robot con todos los componentes y comprobar su funcionamiento.
	\item Diseñar un controlador PID para controlar la velocidad de las ruedas del robot.
	\item Crear una rutina en el microcontrolador que permita la comunicación del robot con los dispositivos del robotario.
	\item Preparar el ordenador a bordo para realizar las tareas de alto nivel del robot y recopilar datos de cámara de a bordo.
\end{enumerate}
\subsection{Construcción de copias del primer robot}
Construir más copias del primer robot. Comprobar las discrepancias que tiene un robot respecto al otro.
\subsection{Configuración y desarrollo de red de comunicación}
El siguiente paso es configurar la red de comunicación, la cual es necesariamente inalámbrica. Se crea un servidor que haga de central, recopile datos de los robots implicados en el Robotario, como posición y velocidad. Además también les manda las instrucciones necesarias para desarrollar los algoritmos.
\subsection{Localización de los robots}
Con los robots configurados, se crea un sistema de localización para poder identificar a los robots y situarlos en el Robotario.

\subsection{Realización de pruebas de validación}
Se realizan diversas pruebas en el robotario para comprobar el funcionamiento del mismo. Se evalúan los datos adquiridos de las pruebas, que permiten conocer las limitaciones del Robotario.


\subsection{Planificación de proyecto}

\begin{figure}[!hp]

	\includegraphics[width=01\linewidth, height=0.75\textheight]{gantt2}
	\caption{diagrama de Gantt}
	\label{fig:gantt}
\end{figure}

En la Figura ~\ref{fig:gantt} se tiene un diagrama de Gantt donde se muestra las fases del proyecto y la duración de las mismas. La duración estimada es de 300 horas repartidas en las semanas mostradas en la Figura~\ref{fig:gantt}. En la fase de montaje de robot se tiene el apartado de montaje de más robots. El numero de robots a montar dependerá de la dificultad o tiempo que requiera y del espacio que se disponga del Robotario. Se tiene planificado hacer una prueba en el Robotario, pero se harán más pruebas si se dispone de más tiempo.
\newpage