\section{Metodología}\label{ch:metodologia}
Para poder llevar a cabo el robotario, se hace un estudio de los ejemplos ya existentes y se establece un método de trabajo.
\subsection{Selección de componentes}
Una vez que se ha recopilado información sobre lo necesario para elaborar el robotario se hace una lista de componentes a nivel de hardware y software.
	
\begin{center}
		\begin{enumerate}
		\item Robot móvil.
		\item Router.
		\item Ordenador que hace de servidor.
		\item Placa con microcontrolador(Arduino).
		\item Cámara webcam.
		\item Cámara a bordo de robot.
		\item Ordenador de placa simple(RaspberryPi).
		\item Selección de software para visión por computador.
		\item Selección de lenguaje de programación.
	\end{enumerate}
\end{center}
\subsection{Construcción de un primer robot}
Lo primero que se hace es evaluar el robot adquirido, como se comento antes, el robot cuenta con dos motores DC, se comprueba la zona de operación de los motores y se estima la velocidad a la cual pueden ir.
	\subsubsection{Prueba de encoders y estimación de la velocidad}
	El siguiente paso es verificar que los encoders ópticos adquiridos definen de manera adecuada los pulsos de salida y que la placa de Arduino cuenta de manera correcta los pulsos, para ello se configura la placa Arduino para que reaccione mediante interrupciones a los pulsos de voltaje proporcionados por el encoder óptico, es decir se provoca una interrupción cuando se tiene un nivel bajo de voltaje después de haber tenido uno alto y se crea una rutina sencilla que mide el tiempo entre interrupciones y cuenta las interrupciones que se producen, de esta manera se cuentan los pulsos.
	\begin{itemize} 
		{
		\item Se conecta la salida digital del encoder a la placa arduino y también a un osciloscopio de esta manera se puede comparar la lecturas que realiza arduino con las medidas reales de pulsos. Se obtiene que el microcontrolador de Arduino lee mas pulsos de los reales, esto es un problema pues no se puede estimar la posición ni la velocidad de manera correcta.\\
		\item Se diseña un filtro paso-bajo rc y se vuelve a comprobar la medición de los pulsos. Se encuentra un resultado favorable respecto al caso anterior, pero sigue habiendo errores en la medida que realiza el microcontrolador de Arduino.
		\item Además del filtro analógico, se diseña un filtro digital el cual consiste en un filtro paso bajo y un filtro de media móvil. Se obtiene un buen resultado, se tiene un error del 2\% en la cuenta de pulsos que realiza la placa de arduino.
		}
	\end{itemize}
	\subsubsection{Prueba de motor}
	
	 Se monta el robot y se ensambla el módulo L298N que incorpora dos puente H y un regulador de tensión. Se conecta la batería al módulo para alimentar los motores y la placa de Arduino mediante el regulador de tensión que proporciona 5 voltios a la salida, a su vez se conectan las salidas de la placa de arduino que proporcionan las señales PWM que se explican en el capitulo ~\ref{ch:ControlMotor}, para el puente H de cada motor y poder gobernar la velocidad a la cual se quiere que vaya ambos motores DC mediante la regulación del voltaje de entrada.\\
	Se encuentra con un problema y es que el puente H proporciona al motor un voltaje máximo de 9V procedente de la batería, haciendo que los motores tengan una velocidad máxima de 30 $rad/s$ y una velocidad mínima de 8.5 $rad/s$. Esto no interesa pues el espacio que se tiene es limitado y una velocidad alta del robot provoca que en un espacio reducido no se pueda apreciar los movimientos que realiza y no se pueda analizar los resultados.\\
	Se reduce el voltaje que alimenta el módulo L298n mediante un conversor DC-DC. Se regula para que proporcione un voltaje de alimentación de 6.5V. Al reducir el voltaje se consigue tener mas niveles de voltajes en rangos menores, debido a la reducción del voltaje se tiene una velocidad menor respecto al caso anterior, siendo la mínima velocidad 7.5$rad/s$ y la máxima 15 $rad/s$ con rozamiento. Se reduce la velocidad mínima porque se ha logrado una mayor resolución en rangos menores de voltaje.
	\subsubsection{programación de arduino y Comprobación del robot montado}
	
	 se crea el programa de control de velocidad y  de la comunicación entre dispositivos. La programación de la placa arduino se realiza con el lenguaje C++ y con las librerías que proporciona el entorno Arduino. El control se hace mediante un controlador PID para cada rueda el cual se detalla en el capitulo ~\ref{ch:ControlMotor}. Para verificar que está bien diseñado se realizan varias pruebas haciendo que ambos motores sigan la misma señal de referencia y al seguir la misma señal de referencia el robot debe ir en línea recta, si este se desvía sobrepasando un margen de error aceptable implica que se debe ajustar los parámetros del controlador.\\
	Se encuentran varios problemas.
	\begin{enumerate}
		\item Las ruedas del motor resbalan sobre el suelo donde se realizan las pruebas que son baldosas blancas estándar. Esto dificulta que el controlador actúe de manera correcta.
		\item Se encuentra un problema con la estima de la velocidad, la medida de velocidad oscila respecto al punto de referencia cuando las pruebas se realizan sobre el suelo, pero si las pruebas se realizan con las ruedas al aire se tienen lecturas que llegan a un estado estacionario.
		\item Las ruedas locas del robot, afectan a la orientación del robot debido al torque que se produce sobre ellas y a la fuerza de rozamiento. Esto es un problema que afecta al controlador de velocidad y a la dirección a la cual se quiere ir.
	\end{enumerate}



Para solucionar el primer problema, se elige un suelo de goma sobre el cual se van a realizar las pruebas, de esta manera las ruedas motrices no patinan y el movimiento se estabiliza.\\

Respecto al segundo problema se han localizado dos fuentes de origen, una es mecánica y otra es procedente de ruido eléctrico.\\
El problema mecánico se debe a que los ejes donde se enganchan las ruedas motrices tienen holgura y esto provoca una oscilación sobre el propio eje de la rueda motriz y otro problema es que el eje se dobla ligeramente cuando se posa sobre el suelo, esto último hace que haya mas superficie de contacto en un lado de la rueda que en otro, provocando perturbación en el movimiento. Estos problemas se deben a la calidad de los componentes y puesto que buscar y adquirir un componente adecuado o un mejor chasis del robot aumentaría el presupuesto se deja como esta y, se trata el problema mecánico como incertidumbre en la medida.\\
El ruido eléctrico se debe a que se alimenta el circuito de potencia y el de la placa de arduino con la misma batería, y lo único que separa un circuito del otro es el regulador de tensión del LN298n, que al ser de baja calidad, no consigue aislar un circuito del otro. La solución que se hace es colocar un condensador electrolítico, como desacoplo en la entrada de alimentación de la placa de arduino. A pesar de la solución se sigue obteniendo ruido debido a la diafonía que producen los pulsos lógicos de lectura de los encoders, ante esto no se puede hacer nada al menos en la placa de arduino de la que se dispone.\\
Las ruedas locas se cambian por otras ruedas, denominadas ruedas de bola que solucionan el problema.\\
Una vez que se tiene ajustado el robot, se construyen 2 robots mas.
\subsection{Configuración y desarrollo de red de comunicación}
El siguiente paso es configurar la red de comunicación, la cual como se ha comentado es necesariamente inalámbrica, por ello se selecciona un router que estaba disponible en el departamento, pero se encuentra un problema y es que este router es muy antiuguo y se debe buscar otro con unos parámetros mejores que el anterior, como el ancho de banda, potencia, procesamiento de la información, etc. En las placas de arduino se crea un programa que permite conectarse a la red LAN. A su vez se crea un pequeño servidor para realizar pruebas de comunicación.Se obtiene de las diversas pruebas realizadas, que se tiene una latencia baja para un dispositivo y el servidor, se realiza una prueba de envio de paquetes con mas arduinos involucrados y se obtiene una latencia mayor entorno a 100 ms, es un factor a tener en cuenta que puede afectar negativamente al robotario.Se elige el protocolo UDP para realizar el envío de información y evitar una latencia mayor y retrasos.\\
 Una vez que se tiene todo funcional se desarrollan funciones, protocolos e instrucciones de comunicación entre las placas de arduino y el servidor de manera que se pueda producir una correcta comunicación. Toda la programación se realiza en C++ y C.\\
 Además de la comunicación inalámbrica se crea una comunicación serie a traves del puerto USB entre arduino y una RaspberryPi que va implementada en el robot. Se usa la misma estructura utilizada para la comunicación WiFi, pero se envia bit a bit.

\subsection{Programación para visión por computador}
Una vez que se tiene la comunicación y el robot preparado se realiza un estudio de la librería de OpenCV en especial de la librería ArUco perteneciente a OpenCV.\\
Estudiada la librería de OpenCV se crea un programa que permite localizar los Markers en el mundo visto por la cámara que en este caso es la cámara logitech, transformando los pixeles de la imagen en coordenadas X,Y,Z con distancia medida en metros, esto realiza con una calibración de la cámara y obteniendo dos matrices, una con los parámetros de distorsión de la cámara y  otra con la relación pixeles metros.\\
 Una vez que se tiene configurado el reconocimiento de markers y su localización en el mundo de la cámara, se procede a hacer lo mismo con la cámara de la raspberryPi de esta manera se tienen calibradas los dos tipos de cámaras.\\
 Finalizada la calibración de las cámaras se colocan en su posición final, la de la raspberryPi se coloca en el robot y la cámara Logitech se coloca encima del lugar que se tiene reservado para el robotario, haciendo de cámara cenital.\\ 
 Por simplicidad solo se usa una cámara cenital.
\subsection{Configuración y programación de RaspberryPI}
Con la cámara de la raspberryPi configurada, se desarrolla un programa que reconoce el entorno y dependiendo del algoritmo a desarrollar, manda instrucciones a la placa de arduino mediante el puerto serie. A su vez se configura la raspberryPi para poder comunicarse con el servidor y poder enviar datos en tiempo real de posición y velocidad, además estos datos se guardan en un archivo en la raspberryPi para poder procesar los datos de los experimentos.

\subsection{Montaje final de Robotario}
 Se coloca una cámara cenital que sirve como sistema de localización global, la cámara se coloca encima del los límites del robotario y lo mas centrada posible.
Por último se instala un suelo de goma en la zona del robotario que maraca los límites. Se hacen pruebas de movimientos con los robots y se detecta que a pesar de ser el suelo de goma, al ser de baja calidad los robots se quedan atrapados en unas zonas, reduciendo su velocidad o en otras zonas sus ruedas motrices resbalan. La solución a futuro es adquirir un suelo de goma duro y liso, similar al que te puedes encontrar en un gimnasio.

En la figura ~\ref{fig:montajefinal} se puede ver como queda el robotario montado, con 3 robots. arriba se puede ver la cámara colocada. En la imagen se aprecia mejor como el espacio es pequeño, por ello se tiene la necesidad de que los robots vayan mas despacio.
\begin{figure}
	\centering
	\includegraphics[width=0.6\linewidth]{MontajeFinal}
	\caption{Montaje final del robotario}
	\label{fig:montajefinal}
\end{figure}