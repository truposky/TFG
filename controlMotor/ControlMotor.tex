\section{Control del Motor DC}\label{ch:ControlMotor}
En esté capítulo se detalla como se ha diseñado el control de velocidad para el motor DC. Como ya se tienen calibrados los encoders para estimar la velocidad de las ruedas, se puede diseñar el control del motor para que siga una referencia de velocidad dada.

\subsection{Cuantización del voltaje}

Para poder tener una velocidad determinada lo que se hace es dar un determinado voltaje al motor DC de cada rueda motriz, esto se hace mediante una señal PWM y un puente H para cada motor, como se ha comentado en la sección el capítulo ~\ref{ch:HardwareYsoftware} el puente H lo proporciona el módulo L298n y la señal PWM la proporciona la placa de Arduino.\\
\begin{figure}[!h]
	\centering
	\includegraphics[width=0.7\linewidth]{dutycicle}
	\caption{Variación del ciclo de trabajo}
	\label{fig:dutycicle}
\end{figure}

Una señal de PWM es una señal periódica con un ancho de pulso ajustable, se usa para transmitir información o para regular la energía transmitida a una carga. En nuestro caso se usa para regular la cantidad de energía que se le entrega al motor y la señal periódica en este caso es una señal cuadrada. La manera en la cual se regula la cantidad de tensión que se le da al motor es cambiando el ciclo del trabajo de la señal cuadrada, es decir se le cambia el tiempo que permanece la señal en una tensión alta o positiva, en la Figura~\ref{fig:dutycicle}, se puede ver como el ancho del pulso de la onda cuadrada cambia con los diferentes valores del ciclo de trabajo y provoca que el voltaje promedio cambie, hay que señalar que solo se cambia el ancho de la onda pero no su periodo pues este es siempre el mismo. El ciclo de trabajo se suele expresa en porcentaje siendo el cálculo $D=\frac{T_{on}}{T}*100$, siendo T el periodo y $T_{on}$ el tiempo que la señal permanece en estado alto.\\
La señal PWM que genera Arduino se genera con un comparador al cual le entran dos señales, una valor fijo ajustable dado por un Timer/contador de 8 bits y una señal triangular generada también por un Timer de 8 bits. El contador de 8 bits va aumentado el valor de 0 a 255, es decir tiene 256 niveles, cuando se supera el valor de 255 el contador vuelve a 0 y repite el ciclo. La frecuencia de la señal PWM para la placa de Arduino nano 33 IoT es de 732Hz\\
\begin{figure}[!h]
	\centering
	\includegraphics[width=0.7\linewidth]{GeneradorDePWM}
	\caption{Ejemplo de señales de entrada al comparador}
	\label{fig:generadordepwm}
\end{figure}
En la figura ~\ref{fig:generadordepwm} se puede ver las señales de entrada al comparador, por un lado se tiene la onda triangular y por el otro el valor fijo ajustable. La salida del comparador es una onda cuadrada, que tiene un valor positivo durante el tiempo que la onda triangular permanece por encima del nivel fijo ajustable. De esta manera se genera una señal PWM en el arduino, esto implica que el voltaje esta cuantizado en 256 niveles y solo se puede variar en valores discretos de estos es decir la resolución de voltaje que se tiene es de $\frac{6.5 V}{256}=0.02V$ es una buena precisión.\\
La señal de PWM que proporciona Arduino es de control y con los transistores que incorpora el integrado, permite regular la tensión al motor, el puente H también permite variar el sentido del giro del motor mediante una señal de control lógica, 0 ó 1.
\begin{figure}[!h]
	\centering
	\includegraphics[width=0.7\linewidth]{controlMotor/PuenteH}
	\caption{Esquema del puente H}
	\label{fig:EsquemaPuenteH}
\end{figure}
Se explica como se gobierna la velocidad del motor y su sentido de giro. Para ello se toma como referencia el puente H de la izquierda, que está compuesto por 4 transistores y 4 puertas AND. La señal de control PWM entra por la entrada EnA, la señal para indicar el giro del motor proviene de in2 e in1, uno debe estar a 0 lógico y el otro a 1 lógico que corresponde  3.3 V, si los dos están al mismo valor el motor no gira, teniendo esto en cuenta cuando la señal procedente de arduino entra en  EnA y a su vez se tiene In1=1 e In2=0, el motor conectado a la salida out1 y out 2 gira en una dirección, es decir se permite el flujo de corriente desde el transistor superior izquierdo hasta el transistor inferior derecho, con voltaje promedio marcado por el ciclo de trabajo de la señal de PWM siendo la alimentación de la señal de potencia la correspondiente a la pata A que está conectado a Vs.


\subsection{Diseño del controlador para la velocidad de las ruedas}\label{ch:controlador}
Para que el robot pueda seguir las instrucciones de manera correcta y los algoritmos se puedan efectuar de manera acertada, el robot debe incorporar un control de la velocidad de las ruedas, el cual se aplica a los dos motores DC de las ruedas motrices. En la figura~\ref{fig:diagramabloque1} se tiene el diagrama de bloque del control que se usa para controlar la velocidad de cada rueda del robot. Cuyas partes son:
\begin{itemize}
	\item $U(s)$. Señal de entrada.
	\item $F$. Control de acción directa.
	\item $PID$. Controlador.
	\item $ZOH$. Retenedor de orden cero, mantiene la señal constante entre muestra y muestra
	\item $G(s)$. Es la planta a controlar. En este caso el motor DC.
	\item $H(s)$. Sensor que estima la velocidad. 
\end{itemize}
\tikzstyle{block} = [%bloque
draw,
minimum width=0.6cm,
minimum height=0.3cm
]
\begin{figure}
\begin{tikzpicture}

% Sum shape
\node[draw,
circle,
minimum size=0.5cm,
fill=Rhodamine!50
] (sum) at (0,0){};

\draw (sum.north east) -- (sum.south west)
(sum.north west) -- (sum.south east);

\draw (sum.north east) -- (sum.south west)
(sum.north west) -- (sum.south east);

\node[left=-3pt] at (sum.center){\tiny $+$};
\node[below=-3pt] at (sum.center){\tiny $-$};



% Sensor block sampler
\node [block,
fill=SeaGreen, 
right=0.5cm of sum
]  (sampler) {   \begin{tikzpicture}
	\draw  ++ (0, 0)
	to [nos,] ++ (0.5,0) ;
	\end{tikzpicture}
};

% Controller
\node [block,
fill=Goldenrod,
right=0.5cm of sampler
]  (controller) {PID(z)};

% Sensor block sampler
\node [block,
fill=SeaGreen, 
right=0.5cm of controller
]  (sampler3) {   \begin{tikzpicture}
	\draw  ++ (0, 0)
	to [nos,] ++ (0.5,0) ;
	\end{tikzpicture}
};

%sum2
\node[draw,
circle,
minimum size=0.5cm,
fill=Rhodamine!50,
right=0.4 cm of sampler3
]  (sum2) {};


\draw (sum2.north east) -- (sum2.south west)
(sum2.north west) -- (sum2.south east);

\draw (sum2.north east) -- (sum2.south west)
(sum2.north west) -- (sum2.south east);

\node[left=-3pt] at (sum2.center){\tiny $+$};
\node[above=-3pt] at (sum2.center){\tiny $+$};

% FeedForward
\node [block,
fill=OrangeRed,
above left= 0.5cm and 0.25cm of controller
]  (feedforward) {$F$};


% Sensor block sampler
\node [block,
fill=SeaGreen, 
right=1cm of sum2
]  (ZOH) {$ZOH$};

% Entrada
\node [block,
fill=BlueGreen,
left=1cm of sum
]  (voltage) {$U(s)$};

% System G(s)
\node [block,
fill=SpringGreen, 
right=0.5cm of ZOH
] (system) {$G(s)$};

% Sensor block H(s)
\node [block,
fill=SeaGreen, 
right= 0.5cm of system
]  (sensor) {$H(s)$};


% Arrows with text label
\draw[-stealth] (sum.east) -- (sampler.west)
node[midway,above]{};

\draw[-stealth] (sampler.east) -- (controller.west)
node[midway,above]{};

\draw[-stealth] (controller.east) -- (sampler3.west) 
node[midway,above]{};

\draw[-stealth] (sampler3.east) -- (sum2.west) 
node[midway,above]{};


\draw[-stealth] (ZOH.east) -- (system.west) 
node[midway,above]{};
\draw[-stealth] (system.east) -- (sensor.west) ;
\draw[-stealth] (sensor.east) -- ++ (0.5,0) 
node[midway](output){}node[midway,above]{$y$};

\draw[-stealth] (output.center) |- (0,-3);
\draw[-stealth] (0,-3) |- (sum.south);


\draw[-stealth] (feedforward.east) -| (sum2.north);


\draw[-stealth] (voltage.east) -- (sum.west);


\draw[-stealth] (voltage.east) |- (feedforward.west);


\draw[-stealth] (sum2.east) |- (ZOH.west);
\end{tikzpicture}
\caption{Diagrama de bloques del control}
\label{fig:diagramabloque1}
\end{figure}
Se realiza una descripción de cada bloque.
\begin{itemize}
	\item $U(s)$ corresponde a la consigna para la velocidad angular de la rueda ($w$ $rad/s$). La entrada se traduce en un valor de PWM que corresponde con un valor de voltaje que está limitado a un valor máximo de 6.5 V.
		\begin{figure}[!hp]
		
		\begin{subfigure}[b]{0.5\linewidth}
			\centering
			\includegraphics[width=1\linewidth]{controlMotor/FeedforwardR1_R}
			\caption{Rueda derecha Robot1\\
				$w=0.687+0.065\cdot pwm$}
			\label{fig:feedforwardR1_R}
			\vspace{4ex}
		\end{subfigure}%%	
		\begin{subfigure}[b]{0.5\linewidth}
			\centering
			\includegraphics[width=1\linewidth]{controlMotor/FeedforwardR1_L}
			\caption{Rueda izquierda Robot1\\
				$w=1.168+0.061\cdot pwm$}
			\label{fig:feedforwardrR1_left}
			\vspace{4ex}
		\end{subfigure}	
		\begin{subfigure}[b]{0.5\linewidth}
			\includegraphics[width=1\linewidth]{controlMotor/FeedforwardR2_R}
			\caption{Rueda derecha Robot2\\
				$w=0.083+0.071\cdot pwm$}
			\label{fig:feedforwardR2_R}
			\vspace{4ex}
		\end{subfigure}
		\begin{subfigure}[b]{0.5\linewidth}
			\includegraphics[width=1\linewidth]{controlMotor/FeedforwardR2_L}
			\caption{Rueda izquierda Robot2\\
				$w=-1.656+0.072\cdot pwm$}
			\label{fig:feedforwardrR2_left}
			\vspace{4ex}
		\end{subfigure}	
		\begin{subfigure}[b]{0.5\linewidth}
			\includegraphics[width=1\linewidth]{controlMotor/FeedforwardR3_R}
			\caption{Rueda derecha Robot3\\
				$w=1.257+0.057\cdot pwm$}
			\label{fig:feedforwardR3_R}
			\vspace{4ex}
		\end{subfigure}
		\begin{subfigure}[b]{0.5\linewidth}
			\includegraphics[width=1\linewidth]{controlMotor/FeedforwardR3_L}
			\caption{Rueda izquierda Robot3\\
				$w=0.332+0.060\cdot pwm$}
			\label{fig:feedforwardrR3_left}
			\vspace{4ex}
		\end{subfigure}
		
		\caption{FeedForward}
		\label{fig:feedforward}
	\end{figure}
	\item $F$ corresponde con un controlador de acción directa que trata de acercar la respuesta del motor al valor de consigna para facilitar la acción del controlador PID. La respuesta no va a ser lineal, debido a la saturación del motor. Para ajustar el valor de su ganancia se ha creado un programa en el microcontrolador, el cual introduce una señal de PWM al puente H y a su vez toma las lecturas proporcionadas por el sensor de velocidad y  se obtiene una tabla de entradas PWM que corresponde a un nivel de voltaje y una salida de $rad/s$ que corresponde al giro de la rueda. Se hace para varios valores y a partir de los datos se hace una regresión lineal de las medidas. En la figura ~\ref{fig:feedforward} se tienen los valores obtenidos para 6 motores que corresponden a 3 robots. La ecuación resultante del ajuste lineal realizado se aplica en el controlador de acción directa de cada motor. Para realizar las medidas se ha tenido en cuenta el rozamiento, esto implica que las pruebas se han realizado con el robot montado y sobre la superficie del Robotario.


Ademas en la figura~\ref{fig:feedforward} se puede apreciar como las curvas de las distintas medidas tienden a la saturación a medida que se aumenta el valor de PWM, este fenómeno se debe a causas mecánicas, como rozamiento, fricción y también a que se está próximo del voltaje máximo que acepta el motor. Otro factor a notar es la diferencia de las medidas de los diferentes robots y en las diferencias de las ruedas del mismo robot. Se puede ver que las medidas dadas por el robot 1 correspondiente a las figuras~\ref{fig:feedforwardR1_R}~\ref{fig:feedforwardrR1_left} son más uniformes y ambas ruedas tienen un comportamiento similar. El robot 3  tiene medidas más dispersas tal y como se aprecian en las figuras~\ref{fig:feedforwardR3_R}~\ref{fig:feedforwardrR3_left}, además se nota en las lecturas de la gráfica que los motores de las ruedas tienen diferentes respuestas y el robot 2 tiene lecturas más uniformes tal y como se aprecia en las figuras~\ref{fig:feedforwardR2_R}~\ref{fig:feedforwardrR2_left} que el robot 3 pero se puede apreciar una diferencia en las medidas obtenidas entre el motor izquierdo y el motor derecho. La respuesta de las gráficas demuestra que cada robot es diferente en cuanto a mediciones y movimiento. También se demuestra con este resultado que los motores de cada robot son muy distintos a pesar de venir del mismo fabricante y tener las mismas características- Otro factor notable en las gráficas para realizar el control de acción directa es que el robot 3 tiene un valor menor de $w$  para un máximo PWM, como el voltaje máximo es el mismo para todos, lo que se deduce de esto es que el robot 3 tiene más rozamiento en las ruedas, esto complica el diseñar un controlador genérico y que sirva para todos los robots, por ello se debe hacer un control de acción directa y un PID distinto para cada robot y rueda del robot .
\newpage
	\item $PID(z)$ corresponde al controlador proporcional, integral y derivativo. El control de acción directa no basta para mantener una velocidad igual a la de referencia, debido a que las condiciones en las que se ha hecho el control de acción directa no son las mismas en todo momento, y además el control de acción directa no responde adecuadamente a las variaciones debidas al ruido o perturbaciones. Es necesario un control realimentado que corrija el error y ajuste la salida al valor deseado. Para ello se usa uno de los controladores más usados y conocidos que es el PID, es un control que es proporcional al error, a su integral y a su derivada temporal. El control proporcional es una ganancia que se multiplica al error para llegar al valor de referencia y minimizar el error, tiene el problema de que se puede quedar en un valor próximo y nunca llegar al valor de consigna. El control integral sirve para integrar el error y como resultado corrige el error al cual no puede llegar el control proporcional. El control derivativo trata de anticiparse a un comportamiento futuro, este último control es muy sensible al ruido, y puesto que el robot tiene mucho ruido en la estimación de la velocidad, se ha decidido no aplicarlo.\\
	
	 Se desconoce como es la planta, se tiene un modelo del motor DC pero el modelo no basta para conocer el comportamiento real. Por ello para la sintonización de los parámetros del PID se usa el método de ziegler-Nichols \cite{feedbacksystems}, es un método que permite sintonizar los parámetros del controlador PID sin tener que conocer la planta, y esto sirve  para una primera aproximación y luego se va ajustando hasta conseguir una respuesta deseada. Ademas de la sintonización de los parámetros del controlador, se implementa un modulo conocido como Anti-Wind-up.
	
	 \subitem El Wind-Up es un problema que se tiene con el control integral, relacionado con el tiempo que tarda el controlador en alcanzar el valor de consigna. Hasta que no se alcance la consigna y el error no cambié de signo el valor de la integral del error sigue creciendo, y puede ocurrir que una vez alcanzado el valor de consigna , el de la integral no sea cero, y el controlador sigue corrigiendo un error que no existe. Esté fenómeno provoca una sobreoscilación indeseada en el sistema, y puede llegar a inestabilizarlo en algunos casos. Debido a esto se propone dentro del control PID el denominado Anti-Wind-Up, básicamente lo que hace es que cuando llega a un valor máximo el integrador deja de acumular valores y cuando se cambia el signo del error se borra la memoria del controlador.
	 
 
	\item $G(s)$ es el motor DC.
	\item $H(s)$ es el sensor, que se refiere al encoder óptico descrito en la sección ~\ref{sc:encoderOptico}
\end{itemize}
En conjunto los bloques forman el lazo de control, que mantiene una velocidad en referencia a la entrada.\\
En el diagrama de bloques anterior no se ha tenido en cuenta las perturbaciones que se obtienen, y debido a que el encoder y los motores no son de buena calidad y a las vibraciones del robot causadas por el movimiento, se tiene una alteración significativa de la lectura de pulsos. En el siguiente apartado se comenta el tratamiento del ruido y su solución.  

\subsection{Sintonización de parámetros PID}
Debido al ruido del sistema, se decide no incluir la parte derivativa del controlador PID. Esto se debe a que la acción derivativa amplifica el ruido. Se podría poner un filtro a la acción derivativa, pero sería complicar el control de manera innecesaria y como se muestra a continuación con un control PI junto al control de acción directa es mas que suficiente.\\
Para la sintonización de los parámetros se recurre al método de Ziglers-Nichols\cite{feedbacksystems}\\
Se toma como ejemplo de sintonización el robot1, procediendo de modo análogo para la sintonización del resto.
Lo primero que se hace es aplicar una ganancia K en el lazo de control, se aumenta dicha ganancia hasta que se llegue a un comportamiento oscilatorio tal y como se aprecia en la figura~\ref{fig:respuestaEscalon}. Con la K que ha causado el comportamiento oscilatorio y el periodo de la oscilación resultante se hallan los parámetros del controlador PI.
\begin{figure}[htbp]

	\begin{subfigure}[b]{0.52\linewidth}
		\centering
	\includegraphics[width=0.95\linewidth]{controlMotor/RespuestaEscalonRD}
	\caption{Respuesta a entrada de referencia W.\\Rueda derecha}
	\label{fig:respuestaescalonrd}
	\end{subfigure}
\quad
	\begin{subfigure}[b]{0.52\linewidth}
		\centering
		\includegraphics[width=1\linewidth]{controlMotor/RespuestaEscalonRI}
		\caption{Respuesta a entrada de referencia W.\\ Rueda izquierda}
		\label{fig:respuestaescalonri}
	\end{subfigure}
	\caption{Respuesta de los motores en el límite de la estabilidad}
	\label{fig:respuestaEscalon}
\end{figure}\\

De la figura~\ref{fig:respuestaEscalon} se obtiene T que es el periodo de la oscilación. Para la rueda derecha $TD=0.59$ que se obtiene tomando el tiempo entre dos máximos, tal y como se aprecia en la figura~\ref{fig:respuestaescalonrd} y para la rueda izquierda se tiene $Ti=0.55$ en la figura~\ref{fig:respuestaescalonri} se tienen los máximos que se han usado. Con estos valores y con k=2, se obtiene los valores del Cuadro~\ref{tab:PID}. Para la ganancia proporcional($K_{p}$) se toma proporciones de la K original, según sea P, PI o PID la proporción cambia. Para solo proporcional se toma 0.5K, para PI se toma 0.4K y para PID se toma 0.6K. En el caso de la ganancia para el control integral($K_{i}$) se toma proporciones del periodo de oscilación, siendo para el control PI 0.8T y para PID 0.5T y por último para la ganancia del control derivativo($K_{d}$) se toma 0.125T como ganancia. \\
\begin{table}[h]
	\begin{center}
	\begin{tabular}{|c|c|c|c|c|c|c|}
		\hline
			& \multicolumn{3}{|c|} {Rueda derecha}& \multicolumn{3}{|c|} {Rueda izquierda}  \\
		\hline
		& $K_{p}$ & $K_{i}$ & $K_{d}$ & $K_{p}$ & $K_{i}$ & $K_{d}$ \\
		\hline
		P & 1 &  &  & 1 &  &  \\
		\hline
		PI & 0.8 & 0.472 &  & 0.8 & 0.44 &  \\
		\hline
		PID & 1.2 &  0.295 & 0.0738 & 1.2 & 0.2750 & 0.0688 \\
		\hline
	\end{tabular}
	\caption{Sintonización de controlador }
	\label{tab:PID} 
	\end{center}
\end{table}
Con las ganancias calculadas en el Cuadro~\ref{tab:PID} se vuelve a realizar una prueba para ver la respuesta de los motores. En la figura~\ref{fig:PI_Nichols} se puede apreciar la respuesta de los motores. Se puede ver que en el motor derecho e izquierdo se tiene una oscilación además de una sobre-elongación grande.


\begin{figure}[h]
	
	\begin{subfigure}[b]{0.52\linewidth}
		\centering
		\includegraphics[width=1\linewidth]{controlMotor/RespuestaEscalonPID_R}
		\caption{PI derecha\\
			Robot1}
		\label{fig:PI_R}
	\end{subfigure}
	\quad
		\begin{subfigure}[b]{0.52\linewidth}
		\centering
		\includegraphics[width=1\linewidth]{controlMotor/RespuestaEscalonPID_I}
		\caption{PI izquierda\\
			Robot1}
		\label{fig:PI_I}
	\end{subfigure}
	\caption{PI Zieglers-Nichols}
	\label{fig:PI_Nichols}
\end{figure}

 Se ajusta manualmente las ganancias del PI y se llega al resultado mostrado en la figura ~\ref{fig:PI_sintonizado}. Esta vez corrige de manera mas eficaz, la sobre-elongación se mantiene, pero dura un breve instante de tiempo, y la oscilación se ha reducido considerablemente, se tiene una pequeña oscilación que se debe a parámetros mecánicos del robot, a la situación de las ruedas, alineación de los ejes, al propio ruido de los encoders y posiblemente a la cuantización del voltaje y no poder llegar a un valor exacto. Aún así el control es bastante efectivo.
 \begin{figure}[!h]
 	
 	\begin{subfigure}[b]{0.49\linewidth}
 		\centering
 		\includegraphics[width=0.8\linewidth]{controlMotor/PI_sintonizado_R}
 		\caption{PI rueda derecha\\
 			Robot1}
 		\label{fig:PIsintonizado_R}
 	\end{subfigure}
 	\quad
 	\begin{subfigure}[b]{0.49\linewidth}
 		\centering
 		\includegraphics[width=0.8\linewidth]{controlMotor/PI_sintonizado_I}
 		\caption{PI rueda izquierda\\
 		Robot1}
 		\label{fig:PIsintonizado_I}
 	\end{subfigure}
 	\caption{PI optimizado}
 	\label{fig:PI_sintonizado}
 \end{figure}
En los otros 2 robots se hace lo mismo, y se llega a tener los siguientes parámetros para los 3 robots que se puede ver en el cuadro ~\ref{tab:PIDSintonizado}, Se observa que los parámetros son muy similares, pero de diferentes valores, a pesar de ser un motor DC de la misma compañía, se ha tenido que variar cada uno individualmente para tener una respuesta aceptable
\begin{table}[!h]
	\begin{center}
		\begin{tabular}{|c|c|c|c|c|c|c|}
			\hline
			& \multicolumn{2}{|c|} {Rueda derecha}& \multicolumn{2}{|c|} {Rueda izquierda}  \\
			
			\hline
			&           kp  &    Ki 	 & kp   &    Ki\\
			\hline
			Robot 1 &  0.1  &  0.05  & 0.15  &   0.035   \\
			\hline 
			Robot 2 &  0.25  &  0.08  & 0.19  &   0.5  \\
			\hline
			Robot 3 &  0.22  &  0.09  & 0.15  &   0.06\\
			\hline
		\end{tabular}
		\caption{Parámetros optimizados }
		\label{tab:PIDSintonizado} 
	\end{center}
\end{table}


\newpage
 