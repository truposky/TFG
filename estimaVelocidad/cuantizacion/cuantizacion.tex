\subsection{Cuantización del voltaje}

Para que el robot adquiera una velocidad determinada lo que se hace es suministrar un voltaje adecuado al motor DC de cada rueda motriz. Esto se hace mediante una señal PWM y un puente H para cada motor.  Como se ha comentado en el capítulo ~\ref{ch:HardwareYsoftware}, el puente H lo proporciona el módulo L298n, la señal PWM  es suministrada por la placa de Arduino. La señal de PWM proporcionada es de control y regula el voltaje de alimentación al motor con ayuda del puente H.\\
\begin{figure}[!h]
	\centering
	\includegraphics[width=0.7\linewidth]{dutycicle}
	\caption{Variación del ciclo de trabajo}
	\label{fig:dutycicle}
\end{figure}

Una señal de PWM es una señal periódica con un ancho de pulso ajustable. Se usa para transmitir información o para regular la energía transmitida a una carga. En nuestro caso se usa para regular la cantidad de energía que se le entrega al motor. La señal periódica en este caso es una señal cuadrada. La manera en la cual se regula la cantidad de tensión que se le da al motor es cambiando el ciclo del trabajo de la señal cuadrada, es decir se cambia el tiempo que permanece la señal en una tensión alta o positiva. En la Figura~\ref{fig:dutycicle}, se puede ver como el ancho del pulso de la onda cuadrada cambia con los diferentes valores del ciclo de trabajo y provoca que el voltaje promedio cambie. Hay que señalar que solo se cambia el ancho de la onda, pero no su periodo pues éste es siempre el mismo. El ciclo de trabajo, D, se suele expresa en porcentaje, $D=\frac{T_{on}}{T}*100$, siendo T el periodo y $T_{on}$ el tiempo que la señal permanece en estado alto.\\
La señal PWM suministrada por Arduino se genera con un comparador al cual le entran dos señales, un valor fijo ajustable dado por un Timer/contador de 8 bits y una señal triangular generada también por un Timer de 8 bits. El contador de 8 bits va aumentado el valor de 0 a 255, es decir tiene 256 niveles, cuando se supera el valor de 255 el contador vuelve a 0 y repite el ciclo. La frecuencia de la señal PWM para la placa de Arduino nano 33 IoT es de 732Hz\\
\begin{figure}[!h]
	\centering
	\includegraphics[width=0.6\linewidth]{GeneradorDePWM}
	\caption{Ejemplo de señales de entrada al comparador}
	\label{fig:generadordepwm}
\end{figure}
En la figura ~\ref{fig:generadordepwm} se puede ver las señales de entrada al comparador, por un lado, se tiene la onda triangular y por el otro el valor fijo ajustable. La salida del comparador es una onda cuadrada, que tiene un valor positivo durante el tiempo que la onda triangular permanece por encima del nivel fijo ajustable. De esta manera se genera una señal PWM en la placa de Arduino, esto implica que el voltaje esta cuantizado en 256 niveles y solo se puede variar en valores discretos de éstos. Es decir, la resolución de voltaje que se tiene es de $\frac{6.5 V}{256}=0.02V$. Es una buena precisión.\\

En la Figura~\ref{fig:EsquemaPuenteH} se tiene el esquemático del integrado L298n. El circuito tiene dos puentes H idénticos. Se toma como referencia el puente H de la izquierda para explicar su funcionamiento y como se regula la alimentación. Está compuesto por 4 transistores y 4 puertas AND. La señal de control PWM procedente de la placa de Arduino entra por la entrada EnA. El puente H se alimenta con una tensión de 6.5V a través de Vs. La señal para indicar el giro del motor proviene de in2 e in1, uno debe estar a 0 lógico y el otro a 1 lógico que corresponde  3.3 V. Estas señales también provienen de la placa de Arduino. Si los dos están al mismo valor el motor no gira. Cuando la señal PWM entra en  EnA y a su vez se tiene In1=1 e In2=0, el motor conectado a la salida out1 y out2 gira en una dirección, es decir se permite el flujo de corriente desde el transistor superior izquierdo hasta el transistor inferior derecho, con un voltaje promedio marcado por el ciclo de trabajo de la señal de PWM.
\begin{figure}[!h]
	\centering
	\includegraphics[width=0.7\linewidth]{controlMotor/PuenteH}
	\caption{Esquema del puente H}
	\label{fig:EsquemaPuenteH}
\end{figure}
