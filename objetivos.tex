\section{Objetivos}
EL objetivo principal es diseñar y crear una infraestructura en la cual, se puedan probar mas allá de la simulación, diversos algoritmos de cooordinación y multiagente, para ello primero se deben cumplir una serie de objetivos específicos, que se enumeran a continuación:
\subsection{Diseñar y montar robot}
Debido a que se quiere verificar como se comportan los algoritmos con agentes reales se deben construir una serie robots que cumplan con unos mínimos.
 
 \begin{enumerate}
 	\item\textit{\textbf{Movilidad}}.El robot debe ser capaz de mantener una velocidad estable, ser capaz de hacer giros de 360º, y tener una deriva perequeña en los movimientos para ser capaz de simular diversos algoritmos.
 	\item\textit{\textbf{Comunicación}}. El robot debe ser capaz de comunicarse con los otros robots y con un servidor. La comunicación debe ser preferiblemente inalámbrica.
 	\item\textit{\textbf{Tamaño}}.El tamaño del robot no debe ser muy grande, para facilitar la manipulación y que el peso del robot no sea un problema, tampoco puede ser muy pequeño, pues un robot de un tamaño muy reducido requeriría componentes electronicos integrados.
 
 \end{enumerate}
\subsection{Red de comunicación inalámbirca}
Se debe diseñar y configurar una red inalámbrica para la comunicación entre los robots y un posible servidor.
\subsection{Sistema de localización}
Se requiere construir o diseñar un sistema de localización global, que identifique mediante algún tipo de identificador como un número, y además sepa localizar los agentes involucrados en el robotario.