\subsection{Experimento II}\label{ch:experimento1}
El segundo experimento se trata de una prueba de apoyo a la investigación, que pone de manifiesto algunas limitaciones del robotario. El objetivo es obtener datos experimentales para validar un algoritmo que estima la posición relativa de un robot en movimiento respecto a otros, empleando medidas locales tomadas desde los robots sin necesidad de posicionamiento global. Este algoritmo está en fase de desarrollo. Desarrollado por Liangming Chen, Hector Garcia de Marina, y Lihua Xie, donde tienen un artículo científico preparando se para publicarlo.\\
Dado un conjunto de robots en movimiento distribuidos en el plano, se puede considerar que en cada instante de tiempo los robots ocupan la posición de los vértices de un polígono. El algoritmo necesita conocer la variación de los ángulos internos del polígono así como la velocidad relativa de los robots para que cada uno determine la posición relativa de sus compañeros.
\subsection{Preparación del experimento}
\begin{figure}[!h]
	\centering
	\includegraphics[width=0.7\linewidth]{Demostracion/trianguloRobotsInicio}
	\caption{Posición inicial de los robots}
	\label{fig:isosceles}
\end{figure}
Inicialmente se planteó un escenario simplificado: Se emplean tres robots (R1,R2,R3) colocados de modo que sus posiciones iniciales formen un triángulo isósceles tal y como se muestra en la Figura~\ref{fig:isosceles}. Cada robot lleva un identificador en su parte frontal, de modo que su posición relativa puede ser estimada por sus compañeros mediante las cámaras que llevan a bordo empleando para ello 