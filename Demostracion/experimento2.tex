\subsection{Prueba I}
La prueba que se realiza se trata de un clásico en la robótica cooperativa conocida en la literatura con el nombre de Rendezvous\cite{10.5555/2901567}. Se trata de una reunión de los robots en un punto del Robotario. Para que pueda ser visible el punto al que se reúnen los robots, se usa un Marker para tal fin.
\subsection{Preparación de la prueba}
Para realizar la prueba primero se debe plantear como orientar los robots al punto deseado y guiarlos. 

\subsubsection{Orientación}

\begin{figure}[!h]
	\centering
	\includegraphics[width=0.75\linewidth]{Demostracion/calculoVector}
	\caption{Cálculo de la orientación del robot}
	\label{fig:calculovector}
\end{figure}
La orientación se realiza mediante un cálculo vectorial. Para explicarlo se toma como ejemplo una posición real de un robot en el Robotario y el destino al cual se quiere ir. Se definen dos vectores. 
\begin{equation}\label{eqn:vectorR2}
V_{R2}=(x1,y1)
\end{equation}
\begin{equation}\label{eqn:vectorP}
V_{p}=(x2,y2)
\end{equation}
El vector $V_{R2}$ corresponde al vector de posición del robot2 y el vector $V_{p}$ a la posición del punto destino. Ambos vectores se definen en el sistema de referencia de la cámara cenital. Para conocer la orientación se calcula un nuevo vector resultante de la diferencia de (~\ref{eqn:vectorP}) y (~\ref{eqn:vectorR2}). En la figura~\ref{fig:calculovector} se tiene una visualización del cálculo que se hace. Con el vector resultante que se define como:
\begin{equation}
V_{D}=(x2-x1,y2-y1)
\end{equation}
Se calcula el ángulo respecto al eje $X$ mediante el arcotangente de las componentes del eje $Y$ respecto a las del eje $X$. Este ángulo se define como $\theta_{2}$. Se define el ángulo $\theta_{1}$ correspondiente a la orientación del robot2. Se obtiene mediante la matriz de rotación proporcionada por la localización del robot tal y como se comentó en la sección~\ref{ch:localizacionMatriz}. Para saber cuánto debe girar el robot para orientarse apuntando al punto destino, se restan ambos ángulos y este se define como el error de orientación.
\begin{equation}
\theta_{e}=\theta_{2} - \theta_{1}
\end{equation}

Para orientar al robot, se debe dar la instrucción de cuanto debe girar. Para ello se recurre a las ecuaciones de navegación desarrolladas en el capítulo~\ref{ch:navegacion} en particular a la ecuación (~\ref{eqn:navegacion}) y se hace girar al robot una cantidad W hasta que el error este en un umbral aceptable. Para corregir el error se recurre a un control PI, proporcional e integral. Se multiplica el error por una constante de proporcionalidad y se integra el error para suavizar el giro que se realiza para orientarse.
\subsubsection{Guiado}
 Una vez que el robot esté orientado se debe iniciar el rumbo al punto deseado y corregir las posibles desviaciones de orientación. Se recurre al mismo cálculo explicado para corregir la orientación. Se fija una velocidad de rumbo fija $V=27.47 cm/s$ y se guía al robot hasta $25 cm$ una distancia de del punto destino, de esta manera se evita que choquen los robots en el punto destino.

\subsection{Resultados}
\begin{figure}[!h]
	\centering
	\includegraphics[width=0.6\linewidth]{Demostracion/navegacion}
	\caption{Navegación de  Robots}
	\label{fig:navegacionExperimento1}
\end{figure}

En la figura~\ref{fig:navegacionExperimento1} se muestra la navegación que han realizado los robots para llegar al punto deseado. El punto inicial viene denotado por un cuadrado. Se observa en la figura~\ref{fig:navegacionExperimento1} que el robot3 (azul), ha recorrido más trayectoria que el robot1 (verde) o el robot2 (rojo). Como el algoritmo es sencillo y solo orienta y corrige el rumbo multiplicando el error por una ganancia determinada, el control es tosco, pero los robots consiguen llegar al destino. En la figura~\ref{fig:navegacionExperimento1} se observan líneas continuas, estas significan que se ha perdido la posición del robot. Esto se debe a que la cámara no es capaz de registrar la posición de los robots en movimiento en cada fotograma. A pesar de esto no supone un problema para el algoritmo, pues los robots consiguen llegar a la posición final. Otro factor a mencionar es la duración de la batería de los robots. Con este experimento la duración aproximada de la batería ha sido de 30 minutos realizando la misma prueba varias veces.


Por último, se muestra en la figura~\ref{fig:experimentoI} una captura de la grabación de la prueba, donde se muestra la posición inicial de los robots, situados en las esquinas del Robotario, y la posición final de los mismo.
\begin{figure}[!h]
	\begin{subfigure}[b]{0.55\linewidth}
		\centering
		\includegraphics[width=0.7\linewidth]{Demostracion/posicionInicial}
		\caption{Posición inicial prueba I}
		\label{fig:posicioninicial}
	\end{subfigure}
	\quad
	\begin{subfigure}[b]{0.55\linewidth}
		\centering
		\includegraphics[width=0.7\linewidth]{Demostracion/posicionFinal}
		\caption{Posición final prueba I}
		\label{fig:posicionfinal}
	\end{subfigure}
	\caption{Prueba I}
	\label{fig:experimentoI}
\end{figure}

\newpage
