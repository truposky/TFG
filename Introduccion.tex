
\section{Introducci\'{o}n}

	En la actualidad existen multitud de aplicaciones roboticas, uno de los campos de interés es el de robots moviles autónomos y por ende el cómo coordinar y controlar los robots para que trabajen de forma colaborativa. Existe un gran interés por parte de empresas privadas e instituciones publicas  en desarrollar algoritmos de control y coordinaci\'{o}n colaborativos para tales propositos.  (rellenar o  cammbiar)
{
\thispagestyle{empty}
\subsection{Motivacion}
Debido a la obligación de comprobar el correcto funcionamiento de los algoritmos de coordinacion, control o calobrativos desarrollados, más allá de las simulaciones, se necesita un entorno que permita observar el comportamiento de los agentes en un entorno lo mas realista posible y por lo tanto surge la necesidad de crear un banco de pruebas en el cual poder estudiar el comportamiento de los respectivos algoritmos.


\subsection{Objetivos}
Para crear un entorno de pruebas se requiere unos determinados componentes, los cuales son: \begin{enumerate}
	\item Construir robots moviles con sensores y capaces de comunicarse para poder comprobar diversos algoritmos.
	\item Diseñar una infraestructura que permita el gobierno de los diferentes robots móviles y una observación de los resultados obtenidos.
	\item Demostrar el funcionamiento del robotario.
\end{enumerate}
                     
\subsection{Metódologia y plan de trabajo}
El proyecto tiene dos partes diferenciadas, una, es la contrucción y el diseño del  control de múltiples robots móviles, para el cual como mas adelante se detallará, consistirá en un robot móvil diferencial gobernado por un microcontrolador Arduino nano 33 IoT, incluye una raspberryPi3 para la visión del entorno y la cual se usara como sensor. Para la comunicación y coordinación de los distintos robots se usara una infraestructura de red de topología jerarquíca, gobernada por un ordenador central, el cual gestionara el movimiento de los robots o recibira los datos de los diferentes robots que existan en la red.

\cleardoublepage
}
