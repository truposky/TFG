

\section{Introducci\'{o}n}\label{ch:introduccion}


 En este proyecto se lleva a cabo el desarrollo de un entorno de pruebas para poder verificar y recopilar datos de algoritmos multiagente que requieran el uso de robots móviles en un entorno realista.\\
  En la actualidad existen multitud de aplicaciones robóticas, en entornos industriales, en tareas que ayudan a los profesionales de la medicina, transporte de mercancías, incluso en la exploración espacial. Uno de los campos de interés de la robótica está relacionado con los sistemas multiagente. Se trata de agentes individuales que colaboran o interactúan entre ellos, utilizados para resolver tareas complejas, o tareas que requerirían mucho tiempo para un agente individual. En la robótica la aplicación de sistemas multiagente tiene actualmente uso en robots móviles que realizan tareas como exploración, transporte de mercancías, o defensa. Debido a que es un área de interés para la robótica, se han creado  multitud de grupos de investigación que abordan el problema de la coordinación y colaboración entre los diferentes robots. Estas investigaciones normalmente dan como resultado algoritmos de coordinación o colaborativos o un conjunto de ambos, que se aplican a los robots.\\
Los escenarios para los que se preparan los algoritmos a veces son muy complejos y, por ello, no se pueden simular en su totalidad. Es preciso verificar el correcto funcionamiento de los algoritmos en un entorno lo más realista posible. Es decir, un entorno experimental con robots reales que permita programar los algoritmos individualmente en cada robot y recopilar datos de su funcionamiento.
\subsection{Motivación}

Debido a la obligación de comprobar el correcto funcionamiento de los algoritmos de coordinación, control o colaborativos desarrollados, más allá de las simulaciones, se necesita un entorno que permita observar el comportamiento de los agentes del modo más realista posible y, por lo tanto, surge la necesidad de crear un banco de pruebas en el cual poder estudiar el comportamiento de los respectivos algoritmos. Con motivo del interés del departamento de arquitectura de computadores y automática de la facultad de C.C. físicas en robótica móvil, se pretende desarrollar un entorno útil para la investigación de dicha robótica y que también sea aplicable a la docencia.

\subsection{Estado del arte}
Respondiendo a la demanda de un entorno de pruebas , varios equipos de investigación han desarrollado entornos de prueba que permiten el análisis de algoritmos colaborativos o algoritmos que requieran el uso robots móviles, uno de los entornos de prueba que permite esta investigación y comprobación de los algoritmos es el Robotarium \cite{pickem2017robotarium}	\cite{8960572} en el cual se basa principalmente este proyecto. El Robotarium es un entorno de prueba desarrollado por un grupo de investigación del instituto  de Tecnología de Georgia en EEUU. El Robotarium fuel el primer entorno de pruebas que se puso a disposición del público, permitiendo que toda persona ya sea investigador, estudiante, trabajador o por simple afición, pudiera probar sus algoritmos en un entorno real de robots, se puede acceder a él en remoto, enviado los algoritmos los cuales son evaluados por la plataforma y te envían los resultados de los experimentos o  se puede ver en directo el experimento.\\

El Robotarium consta de más de un robot móvil, tienen un sistema de localización basado en cámaras cenitales actúan como un sistema de localización global y constan de un sistema de comunicación inalámbrica, además del entorno de pruebas, tienen desarrollado un simulador en matlab o python que permite comprobar que el algoritmo que se quiere implementar funciona en el Robotarium o se puede aplicar, además el coste de acceder es totalmente gratis lo cual lo hace muy atractivo. Actualmente se usa tanto para educación como para investigación.\\

Otro entorno de interés a mencionar es Duckietown \cite{7989179}, es un entorno desarrollado por el instituto Tecnológico de Massachusetts (MIT), y consta también de robots móviles pero a diferencia del Robotarium, los robots de Duckietown tienen cámaras incorporadas que permiten el reconocimiento del entorno. Duckietown esta enfocado en la educación y en la investigación. Tienen un curso enfocado principalmente a Duckietown para estudiantes de grado o postgrados, el cual permite al estudiante familiarizarse con los robots, el control de ellos y el reconocimiento del entorno, además debido a su gran documentación permite que cualquier institución que quiera tenerlo pueda adquirir su hardware y poder reproducir el entorno de Duckietown y realizar investigaciones sobre robots autónomos o inteligencia artificial.\\

Mi proyecto como se ha mencionado esta inspirado en el Robotarium, pero toma como referencia también Duckietown ya que los robots del robotario que es como he denominado mi entorno de pruebas de robots, tienen incorporadas cámaras a bordo que permiten el reconocimiento del entorno y permite desarrollar algoritmos que ademas de una localización global necesiten una localización local y reaccionar ante el entorno del robot. Los robots del robotario no son de un gran presupuesto y no se puede comparar a los entornos mencionados, pero hasta ahora se ha conseguido que permitan desarrollar algoritmos sencillos. Además el espacio del robotario, no es equiparable al Robotarium o Duckietown, pero permite tener al menos más de 2 robots, actualmente se tienen 3 montados con los cuales se pueden verificar diversos algoritmos relacionados con la robótica multi-agente.





\subsection{Objetivos}
El objetivo es crear un entorno de pruebas para poder probar algoritmos colaborativos y de coordinación u otros en los cuales sea necesario el uso de robots móviles. El entorno está basado en los trabajos anteriormente mencionados de Duckietown y Robotarium. Para poder llevarlo acabo se necesita cumplir los siguientes objetivos.
\subsubsection{{\large objetivos específicos }}
\begin{itemize}
	\item Construir y diseñar un robot capaz de realizar los algoritmos requeridos por el robotario.
	\item Diseñar un sistema de control que permita posicionar y mover el robot de acuerdo a las consignas recibidas desde los algoritmos de nivel superior.
	
	\item Configurar la comunicación de los robots, para poder comunicarse con otros robots y un servidor.
	
	\item Diseñar y configurar una red inalámbrica para la comunicación entre los robots y un servidor.
	
	\item Crear y configurar un servidor que estime la posición de los robots y se comunique con los diferentes robots del robotario.
	
	\item Construir y diseñar un sistema de localización, que identifique y localice a los robots involucrados en los experimentos.
	
	\item Llevar a cabo un experimento para demostrar el correcto funcionamiento del robotario.
\end{itemize}





