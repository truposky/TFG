

\section{Introducci\'{o}n}\label{ch:introduccion}


En la actualidad existen multitud de aplicaciones robóticas, en entornos industriales, en tareas que ayudan a los profesionales de la medicina, transporte de mercancías, incluso en la exploración espacial. Uno de los campos de interés de la robótica está relacionado con los sistemas multiagente, se trata de agentes individuales que colaboran entre ellos o interactúan entre ellos, utilizados para resolver tareas complejas, o tareas que requeriría mucho tiempo para un agente individual. En la robótica la aplicación de sistemas multiagente tiene actualmente uso en robots móviles que realizan tareas como exploración, transporte de mercancías, o defensa. Debido a que es un área de interés para la robótica, se han creado  multitud de grupos de investigación que abordan el problema de la coordinación y colaboración entre los diferentes robots. Estas investigaciones normalmente dan como resultado algoritmos de coordinación o colaborativos o un conjunto de ambos, que se aplican a los robots.\\
Los escenarios para los que se preparan los algoritmos, a veces son muy complejos y por ello no se puede simular en su totalidad. Debido a esta razón existe una necesidad de verificar el correcto funcionamiento de los algoritmos en un entorno lo más realista posible. Es decir, un entorno experimental con robots reales que premita programar los algoritmos individualmente en cada robot y recopular datos de su funcionamiento.





\subsection{Estado del arte}
Respondiendo a la demanda de un entorno de pruebas , varios equipos de investigación han desarrollado entornos de prueba que permiten el análisis de algoritmos colaborativos o algoritmos que requieran el uso robots móviles, uno de los entornos de prueba que permite esta investigación y comprobación de los algoritmos es el Robotarium \cite{pickem2017robotarium}	\cite{8960572} en el cual se basa principalmente este proyecto. El Robotarium es un entorno de prueba desarrollado por un grupo de investigación del instituto  de Tecnología de Georgia en EEUU. El Robotarium fuel el primer entorno de pruebas que se puso a disposición del público, permitiendo que toda persona ya sea investigador, estudiante, trabajador o por simple afición, pudiera probar sus algoritmos en un entorno real de robots, se puede acceder a él en remoto, enviado los algoritmos los cuales son evaluados por la plataforma y te envían los resultados de los experimentos o  se puede ver en directo el experimento.\\

El Robotarium consta de más de un robot móvil, tienen un sistema de localización basado en cámaras cenitales actúan como un sistema de localización global y constan de un sistema de comunicación inalámbrica, además del entorno de pruebas, tienen desarrollado un simulador en matlab o python que permite comprobar que el algoritmo que se quiere implementar funciona en el Robotarium o se puede aplicar, además el coste de acceder es totalmente gratis lo cual lo hace muy atractivo. Actualmente se usa tanto para educación como para investigación.\\

Otro entorno de interés a mencionar es Duckietown \cite{7989179}, es un entorno desarrollado por el instituto Tecnológico de Massachusetts (MIT), y consta también de robots móviles pero a diferencia del Robotarium, los robots de Duckietown tienen cámaras incorporadas que permiten el reconocimiento del entorno. Duckietown esta enfocado en la educación y en la investigación. Tienen un curso enfocado principalmente a Duckietown para estudiantes de grado o postgrados, el cual permite al estudiante familiarizarse con los robots, el control de ellos y el reconocimiento del entorno, además debido a su gran documentación permite que cualquier institución que quiera tenerlo pueda adquirir su hardware y poder reproducir el entorno de Duckietown y realizar investigaciones sobre robots autónomos o inteligencia artificial.\\

Mi proyecto como se ha mencionado esta inspirado en el Robotarium, pero toma como referencia también Duckietown ya que los robots del robotario que es como he denominado mi entorno de pruebas de robots, tienen incorporadas cámaras a bordo que permiten el reconocimiento del entorno y permite desarrollar algoritmos que ademas de una localización global necesiten una localización local y reaccionar ante el entorno del robot. Los robots del robotario no son de un gran presupuesto y no se puede comparar a los entornos mencionados, pero hasta ahora se ha conseguido que permitan desarrollar algoritmos sencillos. Además el espacio del robotario, no es equiparable al Robotarium o Duckietown, pero permite tener al menos más de 2 robots, actualmente se tienen 3 montados con los cuales se pueden verificar diversos algoritmos relacionados con la robótica multi-agente.




\thispagestyle{empty}
\subsection{Motivación}
Debido a la obligación de comprobar el correcto funcionamiento de los algoritmos de coordinación, control o colaborativos desarrollados, más allá de las simulaciones, se necesita un entorno que permita observar el comportamiento de los agentes en un entorno lo mas realista posible y por lo tanto surge la necesidad de crear un banco de pruebas en el cual poder estudiar el comportamiento de los respectivos algoritmos. Con motivo del interés del departamento de arquitectura de computadores y automática de la facultad de C.C. físicas en robótica móvil, se pretende desarrollar un entorno útil para la investigación de dicha robótica y que también sea aplicable a la docencia.

\subsection{Objetivos}
El objetivo es crear un entorno de pruebas para poder probar algoritmos colaborativos y de coordinación u otros en los cuales sea necesario el uso de robots móviles. El entorno está basado en los trabajos anteriormente mencionados de Duckietown y Robotarium. Para poder llevarlo acabo se necesita cumplir los siguientes objetivos.
\subsubsection{{\large objetivos específicos }}
\begin{itemize}
	\item Construir y diseñar un robot capaz de realizar los algoritmos requeridos por el robotario.
	\item Diseñar un sistema de control que permita posicionar y mover el robot de acuerdo a las consignas recibidas desde los algoritmos de nivel superior.
	
	\item Configurar la comunicación de los robots, para poder comunicarse con otros robots y un servidor.
	
	\item Diseñar y configurar una red inalámbrica para la comunicación entre los robots y un servidor.
	
	\item Crear y configurar un servidor que estime la posición de los robots y se comunique con los diferentes robots del robotario.
	
	\item Construir y diseñar un sistema de localización, que identifique y localice a los robots involucrados en los experimentos.
	
	\item Llevar a cabo un experimento para demostrar el correcto funcionamiento del robotario.
\end{itemize}





%ya se ha hablado de robotario
\section{Hardware y software}\label{ch:HardwareYsoftware}
A continuación se incluye una breve descripción de los componentes empleados en el desarrollo del robotario.
\subsection{Robot móvil}
El robot móvil del robotario cuenta con varios componentes.
\subsubsection{Chasis del robot}


Se ha elegido un robot móvil de bajo presupuesto comprado en la tienda de Aliexpres\cite{robotAli} pero con piezas simples y comunes, que se pueden reponer de manera sencilla si alguna se estropea, el chasis del robot incluye dos motores DC dos ruedas enganchadas a los motores, dos ruedas locas y consta de dos plantas. Los motores tienen un rango de funcionamiento de 2.5V a 9 V según los datos del fabricante.
\begin{figure}[!h]
	\centering
	\includegraphics[width=0.2\linewidth]{RobotAliexpres}
	\caption{ chasis robot móvil}
	\label{fig:robotaliexpres}
\end{figure}

\subsubsection{Arduino nano 33 IoT}

Para tener un control sobre el robot, obtener datos de los sensores y poder establecer la comunicación del robot con el servidor y los demás robots se usa una placa Arduino\cite{arduinoTienda}, Arduino es una compañía de software y hardware libre que desarrolla y diseña placas con microcontroladores, la placa Arduino adquirida cuenta con un microprocesador Arm Cortex-M0 32-bit SAMD2 además tiene incorporado una antena que permite la comunicación WiFi y Bluetooth, se ha adquirido en la tienda oficial de Arduino. La paca cuenta con una IMU que incorpora acelerómetro y giroscopo, en este proyecto no se va usar este módulo pero es de utilidad para futuras ampliaciones.
\begin{figure}[h]
	\centering
	\includegraphics[width=0.3\linewidth]{Arduinonano33}
	\caption{Placa de Arduino nano 33 IoT}
	\label{fig:arduinonano33}
\end{figure}
\subsubsection{integrado L298n, puente H}
\begin{figure}[h]
	\centering
	\includegraphics[width=0.25\linewidth]{PuenteH}
	\caption{L298n}
	\label{fig:puenteH}
\end{figure}
Para poder gobernar los motores correctamente se usa un puente H\cite{Hhbridge} para cada motor. En este caso se usa un integrado el cual incorpora 2 puentes H y un regulador de tensión que proporciona 5 voltios a la salida y se aprovecha para alimentar la placa de arduino ya que es una tensión estable.El puente H es un circuito formado por transistores y diodos que es de uso común en el gobierno de motores o de circuitos de potencia.El integrado Ln298n cuenta con  un rango de voltaje de entrada de 5V-35V, se alimenta con una batería de tensión nominal de 9.9 voltios. El puente H se alimenta con la tensión de entrada y se regula la alimentación que le llega al motor mediante una señal PWM procedente de la placa arduino. se detalla el funcionamiento de la  señal PWM en conjunto con el puente H en el capítulo ~\ref{ch:ControlMotor} correspondiente al control del motor.
\subsubsection{Batería}
\begin{figure}[h]
	\centering
	\includegraphics[width=0.3\linewidth]{ConstruccionRObot/bat}
	\caption{batería}
	\label{fig:bateria}
\end{figure}
	Para alimentar el robot se usa una batería LiFePo de 3 celdas de 3.3 V cada una en total se tiene 2200mAh y 9.9V. Se usan dos batería en paralelo para poder tener una mayor intensidad y tener una duración mas larga de la batería.

\subsubsection{LM2596 (conversor DC-DC)}
	\begin{figure}[h]
		\centering
	\includegraphics[width=0.25\linewidth]{ConstruccionRObot/StepDown}
	\caption{Step-Down}
	\label{fig:StepDown}
	\end{figure}
Se emplean dos conversores DC-DC para bajar el voltaje de salida de la batería, uno es para alimentar el módulo Ln298 y otro es para alimentar un miniordenador a bordo que requiere como máximo 5 V. De esta manera se evita el uso de otra batería.

\subsubsection{Encoder óptico}	%REVISAR LO DEL FILTRO%
	\begin{figure}[h]
	\centering
	\includegraphics[width=0.25\linewidth]{ConstruccionRObot/Encoder}
	\caption{Encoder óptico}
	\label{fig:encoder}
\end{figure} 
	Una de las partes fundamentales del robot es el encoder, sirve para poder estimar la velocidad a la cual gira cada rueda y poder también estimar la distancia avanzada. El encoder óptico, es el modelo FC-03 , en la Figura ~\ref{fig:encoder} se puede ver como es el componente electrónico. Revisando la hoja de características se sabe que tiene un LM393 el cual es un circuito integrado formado por dos comparadores, también lleva un optointerruptor,el cual consta de un diodo led y un transistor con la base expuesta, si este recibe luz, genera un pulso bajo de tensión , se produce un corto y si no detecta luz genera un pulso alto de tensión, deja el circuito abierto . Mediante el uso de un comparador se obtiene un tren de pulsos cuadrados cuando el optointerruptor es activado o desactivado.\\ 

El encoder además tiene dos salidas, una analógica y otra digital, esta última es la que interesa debido a que es más útil para el microcontrolador leer entradas discretas que equivalen a un 1 cuando se lee un voltaje y un 0 cuando no hay voltaje o esta por debajo de un umbral establecido. El voltaje de funcionamiento del sensor puede ser de 3.3V a 5V, como el microcontrolador solo soporta 3.3V será esta la alimentación que se proporcionará.\\ 
	Para contar pulsos de luz se usa una rueda ranurada que está enganchada al eje de la rueda,y  consta de 20 ranuras esto implica que se tiene una resolución de N=20 y se tiene un paso angular para la rueda de  $\Delta\theta=\frac{2\pi}{20} $ radianes.

\subsubsection{RaspberryPi y Picam}
	\begin{figure}[h]
	\centering
	\includegraphics[width=0.25\linewidth]{raspberry_cam}
	\caption{RaspberryPi conectada con la cámara}
	\label{}
\end{figure}
	Se tiene una placa RaspberryPi4 montada en el robot para poder procesar imágenes adquiridas por una cámara Picam que va conectada a la RaspberryPi.Se alimenta directamente de la batería usando un conversor DC-DC para adaptar el voltaje a 5 V y no tener que usar una segunda batería para alimentar la RaspberryPi. La cámara conectada a la raspberryPi permite obtener imágenes con un angular de 160º según especificaciones del fabricante \cite{Picam}. Este modulo no es esencial para el robotario pero lo hace más versátil pudiendo experimentar con algoritmos de localización a bordo, visión artificial,etc.
	
	\subsection{Camara logitech C920}
		\begin{figure}[!h]
		\centering
		\includegraphics[width=0.1\linewidth]{Camaralogi}
		\caption{Cámara Logitech}
		\label{fig:camaralogi}
	\end{figure}

Se tienen 3 cámaras logitech que se usan para la localización de los robots. En el actual montaje del robotario solo se usa una cámara. La conexión se realiza mediante USB al servidor.
\subsection{OpenCV}
	\begin{figure}[h]
	\centering
	\includegraphics[width=0.3\linewidth]{EjemploDeMarker}
	\caption{sacada de la documentación, ejemplo de un marker}
	\label{fig:ejemplodemarker}
\end{figure}

Para la visión por computador se usa una librería denominada OpenCV \cite{opencv_library} la cual está desarrollada en dos lenguajes de programación c++ y Python. Permite reconocer el entorno con una cámara y tratar las imágenes para desarrollar la aplicación que se desee. En particular yo uso la librería ArUco\cite{ArUco1}\cite{ArUco2} de OpenCV, que permite el reconocimiento de un marker de tamaño 4x4 formado por 16 bits.

	\subsection{Suelo de goma}	
Se usa un suelo de goma que permite que los robots no deslicen al desplazarse, también sirve para delimitar el área del robotario.
\subsection{Router}
El Router permite la comunicación de los distintos dispositivos de la red, es un Router convencional de la marca ZTE. Sus características son las siguientes:
\begin{itemize}
	\item Wireless: en la banda de 2.4GHz se tiene una velocidad de hasta 450Mbps. En la banda de 5GHz se tiene una velocidad de hasta 1300Mbps.
	\item Se tiene un ancho de banda inalámbrico máximo de 1750Mbps.
	\item Potencia máxima en 2.4GHz de 20dBm.
	\item Potencia máxima en 5 GHz de 23dBm y en la banda de 547-5725MHz la potencia máxima es de 30 dBm.
\end{itemize}
Según un análisis realizado por Red zone \cite{router} el router es mas que suficiente para la situación actual de robotario que cuenta con 3 robots contruidos y pronto serán 5. Se entra en más profundidad en el análisis de las características de la red en el capítulo ~\ref{ch:RedLan}.
\newpage
\subsection{Coste del material}
Se muestra el Cuadro ~\ref{tab:MatRobot} con el precio de los componentes de un robot.
\begin{table}[h]
\begin{center}
	\begin{tabular}{| c | c |}
		\hline
		Material Robot & Precio(\euro) \\ 
		\hline
		Arduino nano 33 IoT & 16.00 \\
		\hline
		Chasis robot & 10.47 \\
		\hline
		Ln298(puente H) & 2.79 \\
		\hline
		step Down(x2) & 1.66 \\
		\hline
		Encoder óptico & 2.30 \\
		\hline
		Placa para soldar circuito & 5.05 \\
		\hline
		Ruedas de bola(x2) & 2.74 \\
		\hline
		switch & 0.82 \\
		\hline
		Bateria & 18.90 \\
		\hline
		Condensador de desacoplo & 0.16 \\
		\hline
		Condensador Filtro encoder & 0.06 \\
		\hline
		RaspberryPi 4, 2GB & 57.00\\
		\hline
		Picam & 17.04\\
		\hline
		TOTAL & 134.99 \\
		\hline
	\end{tabular} 
	\caption{Material robot}
	\label{tab:MatRobot} 
\end{center}
\end{table}
El componente más caro del robot es la raspberryPi. Por 134.99\euro se tiene un robot que permite desarrollar diversos algoritmos en el entorno de pruebas,puede comunicarse y reconocer el entorno. Además el robotario puede funcionar sin la raspberryPi y la cámara de Picam, pero esto limita el tipo de experimentos realizables, eso si limitaría el numero de experimentos a realizar. En total el coste de los robots del robotario, con 5 robots adquiridos de los cuales 2 tiene RaspberryPi es de 452.83 \euro.\\
El material restante es el necesario para la comunicación y para que los robots se muevan de manera adecuada es el mostrado en el Cuadro ~\ref{tab:MatRobotario}	
\begin{table}[h]
	\begin{center}
		\begin{tabular}{| c | c |}
			\hline
			Material Robotario & Precio(\euro) \\ 
			\hline
			Cámara logitech c920 & 54.00 \\
			\hline
			Suelo de goma & 20.00 \\
			\hline
			Router & 35.00 \\
			\hline
			TOTAL & 109.00 \\
			\hline
		\end{tabular}
	\caption{Material robotario}
	\label{tab:MatRobotario}  
	\end{center}
\end{table}
\\En total por 561.83 \euro se tiene un robotario funcional.