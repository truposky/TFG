

\section{Introducci\'{o}n}


En la actualidad existen multitud de aplicaciones robóticas, en entornos industriales, en tareas que ayudan a los profesionales de la medicina, transporte de mercancias, incluso en la exploración espacial. Uno de los campos de interés de la robotica está relacionado con los sistemas multiagente, se trata de agentes individuales que colaboran entre ellos o interactúan entre ellos. Estos sistemas multiagente son utilizados para resolver tareas complejas, o tareas que requeriria mucho tiempo para un agente individual. En la robótica la aplicación de sistemas multiagente tiene actualmente uso en robots móviles que realizan tareas como exploración, transporte de mercancias, o usos para la guerra. Debido a que es un area de interés para la robotica, se han creado multitud de investigaciones que abordan el problema de la coordinación y colaboración entre los diferentes robots, estas investigaciones normalmente dan como resultado algoritmos que se aplican a los robots.\\
Los robots implicados en estas tareas suelen ser de un precio elevado dependiendo de las aplicaciones para las cuales se usen, por ello existe una necesidad poder comprobar los algoritmos mas alla de modelos computacionales y simulaciones en un entorno de prueba que no comprometa los prototipos relacioandos con las tareas .\\




\subsection{Estado del arte}
Respondiendo a la demanda de un entorno de pruebas , varios equipos de investigación han desarrollado entornos de prueba que permiten el análisis de algoritmos colaborativos o algoritmos que requieran el uso robots móviles, uno de los entornos de prueba que permite esta investigación y comprobación de los algoritmos es el Robotarium \cite{pickem2017robotarium}	\cite{8960572} en el cual se basa principalmente este proyecto. El Robotarium es un entorno de prueba desarrollado por un grupo de investigación del instituto  de Tecnología de Georgia en EEUU. El Robotarium fuel el primer entorno de pruebas que se puso a disposición del público, permitiendo que toda persona ya sea investigador, estudiante, trabajador o por simple afición, pudiera probar sus algoritmos en un entorno real de robots, se puede acceder a él en remoto, enviado los algoritmos los cuales son evaluados por la plataforma y te envían los resultados de los experimentos o  se puede ver en directo el experimento.\\

El Robotarium consta de más de un robot móvil, tienen un sistema de localización basado en cámaras cenitales actúan como un sistema de localización global y constan de un sistema de comunicación inalámbrica, además del entorno de pruebas, tienen desarrollado un simulador en matlab o python que permite comprobar que el algoritmo que se quiere implementar funciona en el Robotarium o se puede aplicar, además el coste de acceder es totalmente gratis lo cual lo hace muy atractivo. Actualmente se usa tanto para educación como para investigación.\\

Otro entorno de interés a mencionar es Duckietown \cite{7989179}, es un entorno desarrollado por el instituto Tecnológico de Massachusetts (MIT), y consta también de robots móviles pero a diferencia del Robotarium, los robots de Duckietown tienen cámaras incorporadas que permiten el reconocimiento del entorno. Duckietown esta enfocado en la educación y en la investigación. Tienen un curso enfocado principalmente a Duckietown para estudiantes de grado o postgrados, el cual permite al estudiante familiarizarse con los robots, el control de ellos y el reconocimiento del entorno, además debido a su gran documentación permite que cualquier institución que quiera tenerlo pueda adquirir su hardware y poder reproducir el entorno de Duckietown y realizar investigaciones sobre robots autónomos o inteligencia artificial.\\

Mi proyecto como se ha mencionado esta inspirado en el Robotarium, pero toma como referencia también Duckietown ya que los robots del robotario que es como he denominado mi entorno de pruebas de robots, tienen incorporadas cámaras a bordo que permiten el reconocimiento del entorno y permite desarrollar algoritmos que ademas de una localización global necesiten una localización local y reaccionar ante el entorno del robot. Los robots del robotario no son de un gran presupuesto y no se puede comparar a los entornos mencionados, pero hasta ahora se ha conseguido que permitan desarrollar algoritmos sencillos. Además el espacio del robotario, no es equiparable al Robotarium o Duckietown, pero permite tener al menos más de 2 robots, actualmente se tienen 3 montados con los cuales se pueden verificar diversos algoritmos relacionados con la robótica multi-agente.




\thispagestyle{empty}
\subsection{Motivacion}
Debido a la obligación de comprobar el correcto funcionamiento de los algoritmos de coordinacion, control o calobrativos desarrollados, más allá de las simulaciones, se necesita un entorno que permita observar el comportamiento de los agentes en un entorno lo mas realista posible y por lo tanto surge la necesidad de crear un banco de pruebas en el cual poder estudiar el comportamiento de los respectivos algoritmos.

\subsection{Objetivos}
El objetivo es crear un entorno de pruebas para poder probar algoritmos colaborativos y de coordinación u otros en los cuales sea necesario el uso de robots móviles. El entorno está basado en los trabajos anteriormente mencionados de Duckietown y Robotarium. Para poder llevarlo acabo se necesita cumplir los siguientes objetivos.
\subsubsection{{\large objetivos específicos }}
\begin{itemize}
	\item Construir y diseñar un robot capaz de realizar los algoritmos requeridos por el robotario.
	\item Diseñar un sistema de control que permita posicionar y mover el robot de acuerdo a las consignas recibidas desde los algoritmos de nivel superior.
	
	\item Configurar la comunicación de los robots, para poder comunicarse con otros robots y un servidor.
	
	\item Diseñar y configurar una red inalámbrica para la comunicación entre los robots y un servidor.
	
	\item Crear y configurar un servidor que estime la posición de los robots y se comunique con los diferentes robots del robotario.
	
	\item Construir y diseñar un sistema de localización, que identifique y localice a los robots involucrados en los experimentos.
	
	\item Llevar a cabo un experimento para demostrar el correcto funcionamiento del robotario.
\end{itemize}





%ya se ha hablado de robotario
\section{Hardware y software}
El robotario se ha podido desarrollar con los siguientes componentes.
\subsection{Robot móvil}

\begin{figure}[h]
	\centering
	\includegraphics[width=0.3\linewidth]{RobotAliexpres}
	\caption{Robot móvil}
	\label{fig:robotaliexpres}
\end{figure}

Se ha elegido un robot móvil de bajo presupuesto comprado en la pagina de Aliexpres\cite{robotAli} pero con piezas simples que se pueden reponer de manera sencilla si alguna se estropea, el chasis del robot incluye dos motores DC dos ruedas enganchadas a los motores, dos ruedas locas y consta de dos plantas.

\begin{figure}
	\centering
	\includegraphics[width=0.3\linewidth]{Arduinonano33}
	\caption{Placa de Arduino nano 33 IoT}
	\label{fig:arduinonano33}
\end{figure}
\subsection{Arduino nano 33 IoT}

Para tener un control sobre el robot, poder tener sensores en el robot y poder establecer la comunicación del robot con el servidor y los demas robots se usa una placa Arduino\cite{arduinoTienda}, Arduino es una compañia de software y hardware libre que desarrolla y diseña placas con microcontroladores, la placa Arduino adquirida cuenta con un microprocesador Arm® Cortex®-M0 32-bit SAMD2 además tiene incorporado una antena que permite la comunicación WiFi y Bluetooth, se ha adquirido en la tienda oficial de Arduino.
\subsection{L298n}
\begin{figure}[h]
	\centering
	\includegraphics[width=0.25\linewidth]{PuenteH}
	\caption{Puente H}
	\label{fig:puenteH}
\end{figure}
Para poder gobernar los motores correctamente se usa un puente H para cada motor. En este caso se usa un integrado el cual incorpora 2 puentes H y un regulador de tensión que proporciona 5 voltios a la salida y se aprovecha para alimentar la placa con microcontrolador. Se ha adquirido también en la pagina de Aliexpress \cite{puenteHAli}. Cuenta con  un rango de voltaje de entrada de 5V-35V. El puente H se alimenta con la tensión de entrada y se regula la alimetación que le llega al motor mediante una señal PWM\cite{PWM} procedente de la placa arduino. Esta señal PWM que proviene de arduino cuenta con 256 niveles, esto implica se cuantifica el voltaje de entrada en estos 256 voltajes, este numero límitado de niveles de voltaje provoca una limitación en el gobierno de los motores.
\subsection{Bateria}
\begin{figure}[h]
	\centering
	\includegraphics[width=0.3\linewidth]{ConstruccionRObot/bat}
	\caption{bateria}
	\label{fig:bateria}
\end{figure}
	Para alimentar el robot se usa una bateria LiFePo de 3 celdas de 3.3 V cada una en total se tiene 2200mAh. Se usan dos bateria en paralelo para poder tener una mayor intensidad y tener una duración mas larga de la beteria.

\subsection{Step-Down DC-DC}
	\begin{figure}[h]
		\centering
	\includegraphics[width=0.25\linewidth]{ConstruccionRObot/StepDown}
	\caption{Step-Down}
	\label{fig:StepDown}
	\end{figure}
Se tiene un conversor DC-DC para bajar el voltaje de salida de la bateria, en el capitulo ?? se comenta porque.También se ha adquirido en la tienda de Aliexpress[referencia]

\subsection{Encoder óptico}	%REVISAR LO DEL FILTRO%
	\begin{figure}[h]
	\centering
	\includegraphics[width=0.25\linewidth]{ConstruccionRObot/Encoder}
	\caption{Encoder óptico}
	\label{fig:encoder}
\end{figure} 
	Una de las partes fundamentales del robot es el encoder, sirve para poder medir la velocidad a la cual gira cada rueda y poder tambien estimar la distancia avanzada. El encoder óptico, es el modelo FC-03 , en la Figura ~\ref{fig:encoder} se pueder ver su forma. Del datasheet se sabe que tiene un LM393 el cual es un circuito integrado formado por dos comparadores, también lleva un optointerruptor,el cual consta de un diodo led y un transistor con la base expuesta, si este recibe luz, genera un pulso bajo de tensión y si no detecta luz genera un pulso alto de tensión, en conjunto con el comparador se obtendrán pulsos discretizados.\\

El encoder además tiene dos salidas, una analógica y otra digital, esta última es la que interesa debido a que es más útil para el microcontrolador leer entradas discretas. El voltaje de funcionamiento del sensor puede ser de 3.3V a 5V, como el microcontrolador solo soporta 3.3V será esta la alimentación que se proporcionará.\\ 
	Para contar pulsos de luz se usa una rueda ranurada que está enganchada al eje de la rueda,y  consta de 20 ranuras esto implica que se tiene una resolución de N=20 y se divide la rueda en $\Delta\theta=\frac{2\pi}{20} $ radianes.

\subsection{RaspberryPi y Picam}
	\begin{figure}[h]
	\centering
	\includegraphics[width=0.25\linewidth]{raspberry_cam}
	\caption{RaspberryPi conectada con la camara}
	\label{}
\end{figure}
	Se tiene una placa RaspberryPi4 montada en el robot para poder procesar imagenes adquiridas por una camara Picam que va conectada a la RaspberryPi.Se alimenta direactemente de la bateria usando un conversor DC-DC para adaptar el voltaje a 5 V y no tener que usar una segunda bateria para alimentar la RaspberryPi. La camara conectada a la raspberryPi permite obtener imagenes con un angular de 160º segun especificaciones del fabricante \cite{Picam}. Este modulo no es esencial para el robotario pero permite experimentar mas algoritmos debido a la localización con camara a bordo y vision artificial.
	
	\subsection{Camara logitech C920}
		\begin{figure}[h]
		\centering
		\includegraphics[width=0.3\linewidth]{Camaralogi}
		\caption{Cámara Logitech}
		\label{fig:camaralogi}
	\end{figure}

Se tienen 3 cámaras logitech que se usan para la localización de los robots. En el acutal montaje del robotario solo se usa una cámara. La conexión se realiza mediante USB al servidor.
\subsection{OpenCV}
	\begin{figure}[h]
	\centering
	\includegraphics[width=0.3\linewidth]{EjemploDeMarker}
	\caption{sacada de la documentación, ejemplo de un marker}
	\label{fig:ejemplodemarker}
\end{figure}

Para la visión por computador se usa una librería denominada OpenCV \cite{OpenCV} la cual está desarrollada en dos lenguajes de programación c++ y Python. Permite reconocer el entorno con una cámara y tratar las imagenes para desarrollar la aplicación que se desee. En particular yo uso la libreria ArUco\cite{ArUco} de OpenCV, que permite el reconocimiento de un marker de tamaño 4x4 formado por 16 bits.

	\subsection{Suelo de goma}	
Se usa un suelo de goma que permite que los robots no deslicen en el suelo, también sirve para deilimitar el área del robotario.
