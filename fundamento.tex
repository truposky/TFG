\subsection{Estado del arte}
Respondiendo a la demanda de un entorno de pruebas, varios equipos de investigación han desarrollado entornos de prueba que permiten el analisis de algoritmos colaborativos o algoritmos que requieran el uso robots móviles, uno de los entornos de prueba que permite esta investigación y comprobación de los algoritmos es el Robotarium \cite{robotarium}. El Robotarium es un entorno de prueba desarrollado por un grupo de investigacion del instituto  de Tecnología de Georgia en EEUU. El Robotarium fuel el primer entorno de pruebas que se puso a disposición del público, permitiendo que toda persona ya sea investigador, estudiante, trabajador o por simple afición, pudiera probar sus algoritmos en un entorno real de robots.\\
El Robotarium consta de mas de un robot móvil, tienen un sistema de localización basado en camaras cenitales que act´ y constan de un sistema de comunicación, además del entorno de pruebas, tienen desarrollado un simulador en matlab que permite comprobar que el algoritmo que se quiere probar se puede implementar en el Robotarium.\\
Otro entorno de interés a mencionar es Duckietown \cite{duckietown}, es un entorno desarrollado por el instituto Tecnológico de Massachusetts, y consta también de robots móviles pero a diferencia del Robotarium, los robots de Duckietown tienen cámaras incorporadas que permiten el reconocimiento del entorno.



