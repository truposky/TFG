\subsection{Estado del arte}


Respondiendo esta demanda de un entorno de pruebas , varios equipos de investigación han desarrollado entornos de prueba que permiten el análisis de algoritmos colaborativos o algoritmos que requieran el uso de robots móviles, uno de los entornos de prueba que permite esta investigación y comprobación de los algoritmos es el Robotarium \cite{pickem2017robotarium}	\cite{8960572} en el cual se basa principalmente este proyecto. El Robotarium es un entorno de prueba desarrollado por un grupo de investigación del instituto  de Tecnología de Georgia en EEUU. El Robotarium fuel el primer entorno de pruebas que se puso a disposición del público, permitiendo que toda persona ya sea investigador, estudiante, trabajador o por simple afición, pudiera probar sus algoritmos en un entorno real de robots. Se puede acceder a él en remoto, enviado los algoritmos ,estos son evaluados por la plataforma y los resultados obtenidos son enviados de vuelta al desarrollador, Además es posible ver en directo el experimento.\\

El Robotarium dispone de hasta 100 robots móviles de ruedas con control diferencial, que pueden desplazarse sobre una superficie lisa rectangular. Tiene un sistema de cámaras cenitales que actúan como sistema de localización global y posee también un sistema de comunicación inalámbrica, que permite a los robots comunicarse con un servidor fijo.A parte del entorno de pruebas, tienen desarrollado un simulador en matlab o python que permite comprobar que el algoritmo que se quiere implementar funciona en el Robotarium o se puede aplicar.Se puede acceder sin coste alguno, es totalmente gratis lo cual lo hace muy atractivo. Actualmente se usa tanto para educación como para investigación.\\

Otro entorno de interés a mencionar es Duckietown \cite{7989179}, es un entorno desarrollado por el instituto Tecnológico de Massachusetts (MIT), y consta también de robots móviles pero a diferencia del Robotarium, los robots de Duckietown tienen cámaras incorporadas que permiten el reconocimiento del entorno. Duckietown esta enfocado en la educación y en la investigación. Tienen un curso enfocado principalmente a Duckietown para estudiantes de grado o postgrados, el cual permite al estudiante familiarizarse con los robots, el control de ellos y el reconocimiento del entorno, además debido a su gran documentación permite que cualquier institución que quiera tenerlo pueda adquirir su hardware y poder reproducir el entorno de Duckietown y realizar investigaciones sobre robots autónomos o inteligencia artificial.\\

Mi proyecto como se ha mencionado esta inspirado en el Robotarium, pero toma como referencia también Duckietown ya que los robots del robotario que es como he denominado mi entorno de pruebas de robots, tienen incorporadas cámaras a bordo que permiten el reconocimiento del entorno y permite desarrollar algoritmos que ademas de una localización global necesiten una localización local y reaccionar ante el entorno del robot. Los robots del robotario no son de un gran presupuesto y no se puede comparar a los entornos mencionados, pero hasta ahora se ha conseguido que permitan desarrollar algoritmos sencillos. Además el espacio del robotario, no es equiparable al Robotarium o Duckietown, pero permite tener al menos más de 2 robots, actualmente se tienen 3 montados con los cuales se pueden verificar diversos algoritmos relacionados con la robótica multi-agente.



