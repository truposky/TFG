\subsection{Navegación del robot}

\begin{figure}[htbp]
	
	\begin{subfigure}[b]{0.52\linewidth}
		\centering
		\includegraphics[width=0.7\linewidth]{navegacion/RuedaRobotDibujo}
		\caption{Rueda de robot}
		\label{fig:ruedarobotdibujo}
	\end{subfigure}
	\quad
	\begin{subfigure}[b]{0.52\linewidth}
		\centering
		\includegraphics[width=0.7\linewidth]{navegacion/ArcoRecorridoRobot}
		\caption{Arco que recorre robot}
		\label{fig:arcorecorridorobot}
	\end{subfigure}
\end{figure}
Una vez que se ha sintonizado el controlador y que se obtiene un movimiento suave y uniforme, se calculan las ecuaciones de cinemáticas que gobiernan al robot y sirven para establecer la navegación del robot.
Se puede ver en la Figura ~\ref{fig:arcorecorridorobot} la distancia entre las ruedas del robot(L) y el arco que forma el robot al moverse,$S1$ y $S2$.
La rueda izquierda forma un arco de radio x y la rueda derecha un arco de radio $x+l$
	siendo las longitudes de ambos arcos:
	
	\begin{center}

		$S2=x*\Delta\varphi=R\Delta\theta$\\
	$S1=(x+L)*\Delta\varphi=R\Delta\theta$\\
	\end{center}
	Donde  $R\Delta\theta$ es la distancia que avanza el robot y $\Delta\theta=\frac{2\pi}{N}i$ son los pasos de encoder del robot, siendo R el radio de la rueda e i los pulsos de encoder. Se hace la resta de S1 y S2.
	\begin{center}
	 $S1-S2= L\Delta\varphi = R*\Delta\theta_{d}-R\Delta\theta_{i}$\\
	 $\Delta\varphi = \frac{R(\Delta\theta_{d} - \Delta\theta_{i})}{L}$\\
	\end{center}
	 Se deriva respecto al tiempo.\\
	 \begin{center}
	 $ \frac{\Delta\varphi}{\Delta t}=  \frac{R(\Delta\theta_{d} - \Delta\theta_{i})}{L*\Delta t} = \frac{r(w_{d}-w{i})}{L}$\\
	
	\end{center}
 de esta manera se tiene la velocidad angular del robot que depende de la velocidad angular de cada rueda y de la distancia que las separa. Y por último la velocidad lineal del robot viene dada por la siguiente expresión.
 
 \begin{center}
 	$V= \frac{v_{d} + v_{i}}{2} = \frac{R(w_{d}-w{i})}{2}$
 \end{center}
Con las expresiones obtenidas, lo siguiente que se hace es ponerlas en forma matricial para poder pocersarlas en el robot.
\[
\begin{pmatrix}
W\\
V
\end{pmatrix}
= \begin{pmatrix}
\frac{R}{L} & \frac{-R}{L}\\
\frac{R}{2} & \frac{R}{2}
\end{pmatrix}*
\begin{pmatrix}
w_{D}\\
w_{I}
\end{pmatrix}
\]
Las consignas de orden para el robot son $W$ y $V$ y el robot tendra como entrada $w_{D}$ y $w_{I}$, se resuelve la ecuación matricial y se obtiene las velocidades angulares de las ruedas en función de $W$ y $V$.
\[
\begin{pmatrix}
w_{D}\\
w_{I}
\end{pmatrix}
= \begin{pmatrix}
\frac{L}{2R} & \frac{1}{R}\\
\frac{-L}{2R} & \frac{1}{R}
\end{pmatrix}*
\begin{pmatrix}
W\\
V
\end{pmatrix}
\]

Esta matriz se puede implementar en el código del robot y se tiene el gobierno del movimiento del robot.

