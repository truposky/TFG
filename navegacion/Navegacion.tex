\section{Navegación del robot}\label{ch:navegacion}

\begin{figure}[htbp]
	
	\begin{subfigure}[b]{0.4\linewidth}
		\centering
		\includegraphics[width=0.5\linewidth]{navegacion/RuedaRobotDibujo}
		\caption{Rueda de robot}
		\label{fig:ruedarobotdibujo}
	\end{subfigure}
	\quad
	\begin{subfigure}[b]{0.5\linewidth}
		\centering
		\includegraphics[width=0.4\linewidth]{navegacion/ArcoRecorridoRobot}
		\caption{Arco que recorre robot}
		\label{fig:arcorecorridorobot}
	\end{subfigure}
\caption{Parámetros usados para el cálculo de la navegación}
\end{figure}
Una vez que se ha sintonizado el controlador y que se obtiene un movimiento suave y uniforme, se calculan las ecuaciones cinemáticas que gobiernan al robot y sirven para establecer la navegación.
Se puede ver en la figura~\ref{fig:arcorecorridorobot} la distancia entre las ruedas del robot(L) que es de 12.5cm y los arcos S1 y S2 que forman al moverse el robot.
La rueda izquierda forma un arco de radio x y la rueda derecha un arco de radio $x+l$
	siendo las longitudes de ambos arcos:
	
	\begin{center}
	\begin{equation}
	S2=x*\Delta\varphi=R\Delta\theta_{2}
	\end{equation}
	\begin{equation}
	S1=(x+L)*\Delta\varphi=R\Delta\theta_{1}
	\end{equation}
	
	\end{center}
	Donde $R\Delta\theta$ es la distancia que avanza el robot y $\Delta\theta_{1,2}=\frac{2\pi}{N}i$ son los pasos del encoder de cada rueda, siendo, R el radio de la rueda e i los pulsos del encoder. Se hace la resta de S1 y S2.
	
	\begin{equation}
		S1-S2= L\Delta\varphi = R*\Delta\theta_{d}-R\Delta\theta_{i}
		\end{equation}
	\begin{equation}
	\Delta\varphi = \frac{R(\Delta\theta_{d} - \Delta\theta_{i})}{L}
	\end{equation}
	 
	 Se deriva respecto al tiempo.
	 \begin{equation}
	 	W= \frac{\Delta\varphi}{\Delta t}=  \frac{R(\Delta\theta_{d} - \Delta\theta_{i})}{L*\Delta t} = \frac{R(w_{d}-w{i})}{L}
	\end{equation}
 de esta manera se tiene la velocidad angular del robot que depende de la velocidad angular de cada rueda y de la distancia que las separa. Y por último la velocidad lineal del robot viene dada por la siguiente expresión.
 
 \begin{equation}
 	V= \frac{v_{d} + v_{i}}{2} = \frac{R(w_{d} + w{i})}{2}
 \end{equation}
Las expresiones obtenidas pueden representarse en forma matricial como:
\begin{equation}
\label{eqn:navegacion}
\begin{pmatrix}
W\\
V
\end{pmatrix}
= \begin{pmatrix}
\frac{R}{L} & \frac{-R}{L}\\
\frac{R}{2} & \frac{R}{2}
\end{pmatrix}*
\begin{pmatrix}
w_{D}\\
w_{I}
\end{pmatrix}
\end{equation}
Las consignas de orden para el robot son $W$ y $V$ y el robot tendrá como entrada $w_{D}$ y $w_{I}$, se resuelve la ecuación matricial y se obtiene las velocidades angulares de las ruedas en función de $W$ y $V$.
\begin{equation}
\begin{pmatrix}
w_{D}\\
w_{I}
\end{pmatrix}
= \begin{pmatrix}
\frac{L}{2R} & \frac{1}{R}\\
\frac{-L}{2R} & \frac{1}{R}
\end{pmatrix}*
\begin{pmatrix}
W\\
V
\end{pmatrix}
\end{equation}

Esta matriz se puede implementar en el código del robot y se tiene el gobierno del movimiento del robot.


